\documentclass[11pt,a4paper,notitlepage]{article}

\usepackage[utf8]{inputenc}
\usepackage[T1]{fontenc}
\usepackage[german]{babel}
\usepackage{float}
%\usepackage{amsmath}
%\usepackage{amsfonts}
\usepackage{amssymb}
\usepackage{graphicx}
\usepackage{ifthen}
\usepackage{diagbox}
\usepackage[hyphens]{url}
\usepackage{textcomp,gensymb}
\usepackage{makecell}
\usepackage{textpos}
\usepackage{tabularx}
\usepackage{csquotes}
%\usepackage{hyperref}
\usepackage[natbib=true,bibstyle=numeric,backend=bibtexu,citestyle=numeric]{biblatex}
\bibliography{lit.bib} 
\renewcommand\theadalign{cb}
\renewcommand\theadfont{\bfseries}
\renewcommand\theadgape{\Gape[4pt]}

\usepackage{nopageno}

\author{}
\date{}
\title{\name}

% Template für Checkboxen ----
\newcommand{\checkbox}[1]{
\ifx#1\undefined
  $\Box$
\else
  $\boxtimes$  
\fi}

\setlength{\parindent}{0pt}

%----------------------------

\newcommand{\grafischedarstellung}{\jobname_graphical_description.png}

%----------------------------

\newcommand{\umldiagram}{\jobname_uml.png}

%----------------------------

\newcommand{\sequencediagram}{\jobname_sequence.png}

%----------------------------

\newcommand{\solutionimg}{\jobname_solution.png}

%----------------------------

\newcommand{\prototypeimg}{\jobname_prototype.png}

% --------- Glossary
\newcommand{\sen}{Sender}
\newcommand{\rec}{Empfänger}
\newcommand{\recdev}{Empfangsgerät}
\newcommand{\sendev}{Sendegerät}
\newcommand{\data}{Datenobjekt}

\newcommand{\name}{Template Beschreibung}

% -------------------------------
% WAS
% -------------------------------

\newcommand{\desc}{Problem des Nutzers schildern, z.B. wie ein Datensatz von einem Gerät auf ein anderes übertragen werden soll oder wie ein Bildschirminhalt auf ein weiteres Gerät erweitert dargestellt werden soll.}

\newcommand{\solution}{Eine allgemeine, grobe Beschreibung soll dem Leser den ersten Eindruck und Überblick über die Interaktion vermitteln bzw. die Lösung des zuvor beschriebenen Problems umfassen. }

\newcommand{\category}{give}
%\newcommand{\category}{take}
%\newcommand{\category}{exchange}
%\newcommand{\category}{extend}
%\newcommand{\category}{connect}

% -------------------------------
% WIE
% -------------------------------

\newcommand{\useraction}{Handlungsablauf aus der Nutzersicht erläutern. 
Wie sieht der Bewegungsablauf der Interaktion aus?}

\newcommand{\reaction}{Reaktion des Systems bzw. der Geräte erläutern bei Erfolg als auch bei Nicht-Erfolg.
Wie verhält sich das Gerät (z.B. akustisch), wie verhält sich das System (z.B. visuell)?}

\newcommand{\designnotes}{Hinweise für die Gestalter der Interaktion, was der Benutzer beim Ausführen beachten muss, z.B. den zu erbringenden Kraftaufwand bei der Interaktionsausführung oder wann das Gerät/die Geräte eine visuelle Rückmeldung geben sollen.\\ 
Welche Geschwindigkeit muss das Gerät ggf. erreichen?\\
In welche Richtung muss das Gerät gerichtet werden? 
	
}

% -------------------------------
% WANN
% -------------------------------

\newcommand{\validcontext}{Wann kann das Pattern angewendet werden?\\
Beschreibung oder Auflistung von Verwendungsmöglichkeiten (Kontexten), in denen die Interaktion zweckmäßig Gebrauch findet.}

%\newcommand{\simultaneously}{}
%\newcommand{\sequentially}{}

%\newcommand{\online}{}
%\newcommand{\offline}{}

%\newcommand{\private}{}
%\newcommand{\semipublic}{}
%\newcommand{\public}{}
%\newcommand{\stationary}{}
%\newcommand{\onthego}{}

%\newcommand{\leanback}{}
%\newcommand{\leanforward}{}

%\newcommand{\single}{}
%\newcommand{\collaboration}{}

%\newcommand{\smalltask}{}
%\newcommand{\repeatedtask}{}
%\newcommand{\locationbased}{}
%\newcommand{\distraction}{}
%\newcommand{\urgent}{} 

\newcommand{\notvalidcontext}{Wann kann das Pattern nicht angewendet werden?\\
Beschreibung oder Auflistung von Verwendungsmöglichkeiten oder Kontexten, in denen die Interaktion keinen zweckmäßigen Gebrauch findet.}


\newcommand{\devicetabular}{
\begin{tabular}[H]{|c|c|c|c|c|c|}
\hline 
von/nach & Smartwatch & Smartphone & Tablet & Tabletop & Screens \\ 
\hline 
Smartwatch & • & • & • & • & • \\ 
\hline 
Smartphone & • & • & • & • & •\\ 
\hline 
Tablet & • & • & • & • & •\\ 
\hline 
Tabletop & • & • & • & • & •\\ 
\hline
Screens & • & • & • & • & • \\ 
\hline 
\end{tabular} }


%\newcommand{\distanceIntimate}{}
%\newcommand{\distancePersonal}{}
%\newcommand{\distanceSocial}{}
%\newcommand{\distancePublic}{}

% -------------------------------
% WARUM
% -------------------------------

%\newcommand{\established}{}
%\newcommand{\candidate}{}
%\newcommand{\realizable}{}
%\newcommand{\futuristic}{}

\newcommand{\otherpatterns}{Auflistung oder Beschreibung von Pattern, die dem beschriebenen Pattern hinsichtlich der Geste ähneln.}

\newcommand{\stateoftheart}{Verweis auf Projekte/Anwendungen/Produkte, in denen die beschriebene Interaktion bereits genutzt wird.}

%\newcommand{\designprinciples}{}

%\newcommand{\imageschemata}{}
%\newcommand{\imageschemataA}{}
%\newcommand{\imageschemataB}{}
%\newcommand{\imageschemataC}{}

%\newcommand{\realworld}{}
%\newcommand{\realworldA}{}
%\newcommand{\realworldB}{}
%\newcommand{\realworldC}{}
%\newcommand{\realworldD}{}

%\newcommand{\metaphor}{}
\newcommand{\metaphordesc}{Assoziationen/Metaphern zu der beschriebenen Interaktion mit bekannten Gesten bzw. Ursprüngen aus der Natur.}

% -------------------------------
% TECHNISCHES
% -------------------------------

%\newcommand{\technologyObjectA}{}
%\newcommand{\technologyObjectB}{}
%\newcommand{\technologyObjectC}{}
%\newcommand{\technologyObjectD}{}

\newcommand{\technologyObjectDesc}{Technologien unterschieden durch die Entfernung zwischen den Geräten, die die Objekterkennung unterstützen – sofern die Technologie für das beschriebe Pattern von Gebrauch ist.}

%\newcommand{\technologyCommunicationServer}{}
%\newcommand{\technologyCommunicationAdhoc}{}

\newcommand{\technologyCommunicationDesc}{Technologien, die die (Daten-) Kommunikation unterstützen – sofern die Technologie für das beschriebe Pattern von Gebrauch ist.}

%\newcommand{\technologyOrientationAccelerometer}{}
%\newcommand{\technologyOrientationGPS}{}
%\newcommand{\technologyOrientationGyroskop}{}
%\newcommand{\technologyOrientationAnnaeherung}{}
%\newcommand{\technologyOrientationHoehe}{}
%\newcommand{\technologyOrientationBeacons}{}
%\newcommand{\technologyOrientationOther}{}

\newcommand{\technologyOrientationDesc}{Technologien, die die Bestimmung der Bewegung bzw. Orientierung des Gerätes unterstützen – sofern die Technologie für das beschriebe Pattern von Gebrauch ist.}

\newcommand{\prototype}{Verweise, Ideen, Prototypen, Code Snippets und UML-Diagramme sowie weitere technisch hilfreiche Darstellungen können als Ansatz oder Anleitung dienen, um das beschriebene Interaktions-Pattern zu realisieren.}


% -------------------------------
% SONSTIGES
% -------------------------------

\newcommand{\authors}{Auflistung der Person bzw. Personen, die an der Entwicklung des beschriebenen Interaktions-Patterns beteiligt sind.}
\newcommand{\literature}{Sammlung und Auflistung aller Literatur- als auch Videoreferenzen zum beschriebenen Pattern.}
\newcommand{\figures}{Auflistung aller Abbildungsreferenzen zum beschriebenen Pattern.}
\newcommand{\versionhistory}{Versionshistorie des beschriebenen Patterns.}
\newcommand{\comments}{Kommentare zur aktuellen Version.}
\newcommand{\questions}{Offene Fragen zum beschriebenen Pattern, die in einer weiteren Version ggf. beantwortet werden und zur Weiterentwicklung des Patterns beitragen.}


% template inkludieren --------------

\maketitle


\section*{Was}

\subsection*{Problem}
\desc

\subsection*{Lösung}
\solution

\subsection*{Grafische Darstellung}

\begin{figure}[H]
\includegraphics[scale=0.3]{mypicture.png}
\end{figure}


\subsection*{Kategorie}
\ifthenelse{\equal{\category}{give}}{$\boxtimes$}{$\Box$} Give \\
\ifthenelse{\equal{\category}{take}}{$\boxtimes$}{$\Box$} Take \\
\ifthenelse{\equal{\category}{exchange}}{$\boxtimes$}{$\Box$} Exchange \\
\ifthenelse{\equal{\category}{extend}}{$\boxtimes$}{$\Box$} Extend \\
\ifthenelse{\equal{\category}{connect}}{$\boxtimes$}{$\Box$} Connect



\section*{Wie}

\subsection*{Aktion des Benutzers}
\useraction

\subsection*{Reaktion des Sende-und Empfänger-Gerätes}
%\reaction

\subsection*{Hinweise zur Gestaltung der Interaktion}
%\designnotes



\section*{Wann}

\subsection*{Geeigneter Nutzungskontext}

\subsubsection*{Zeit}
\ifthenelse{\equal{\when}{gleichzeitig}}{$\boxtimes$}{$\Box$} gleichzeitige Nutzung von Geräten \\
\ifthenelse{\equal{\when}{aufeinanderfolgend}}{$\boxtimes$}{$\Box$} sequentielle Nutzung von Geräten 

\subsubsection*{Modus}
\ifthenelse{\equal{\mode}{online}}{$\boxtimes$}{$\Box$} online \\
\ifthenelse{\equal{\mode}{offline}}{$\boxtimes$}{$\Box$} offline \\

%\validcontext

\subsection*{Abzuratender Nutzungskontext}
%\notvalidcontext

\subsection*{Geräteklassen}
\begin{tabular}{|c|c|c|c|c|}
\hline 
• & • & Mittel & Riesig & Groß \\ 
\hline 
• & • & • & • & • \\ 
\hline 
• & • & • & • & • \\ 
\hline 
• & • & • & • & • \\ 
\hline 
• & • & • & • & • \\ 
\hline 
\end{tabular} 

\subsection*{Entfernung zwischen Sender- und Empfänger-Gerät}



\section*{Warum}


\subsection*{Displaygrößen}


\subsection*{Analoge Patterns}


\subsection*{State of the Art/Gebrauchshistorie}


\subsection*{Checkliste: Entspricht die Interaktion der Definiton eines "Blended Interaction"?}


\section*{Technisches}

\subsection*{Technologien zur Objekterkennung}


\subsection*{Technologien zur Kommunikation}


\subsection*{Technologien zur Bewegungs-/Orientierungsbestimmung}


\subsection*{Prototyp/Lösungsansatz/Code-Snippets/UML-Diagramm}



\section*{Sonstiges}

\subsection*{Autor/en}

\subsection*{Literaturreferenzen}

\subsection*{Abbildungsverzeichnis}

\subsection*{Versionshistorie}

\subsection*{Kommentare}

\subsection*{Offene Fragen}