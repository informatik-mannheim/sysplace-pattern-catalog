\newglossaryentry{swipe}
{
    name=Swipe,
    description={Ein Swipe ist eine schnelle Wischbewegung eines (oder mehrerer) Fingers über einen Touchscreen um eine Funktion auszuführen}
}

\newglossaryentry{atomareinteraktion}
{
    name=Atomare Interaktion,
    description={Der Begriff Atomare Interaktion wurde von Dan Saffers Konzept der Microinteractions abgeleitet und beschreibt eine noch feinere Betrachtung einer Nutzerinteraktion},
    plural={Atomare Interaktionen}
}

\newglossaryentry{microinteraction}
{
    name=Microinteraction,
    description={Microinteractions sind die funktionalen, interaktiven Details eines jeden Produkts. Eine Microinteraction besteht aus vier Elementen: Einem Auslöser („Trigger“), Regeln („Rules“), Feedback und Schleifen und Modi („Loops and Modes“)}
}

\newglossaryentry{bump}
{
	name=Bump,
	description={Engl. für zusammenstoßen}
}

\newglossaryentry{einfachegeste}
{
	name=einfache Geste,
	description={Als einfache Gesten werden Interaktionen bezeichnet, deren Bewegungsmuster durch Sensoren auf einem einzelnen Gerät erkannt werden können, ohne dass eine Kommunikation mit anderen Geräten oder Eingaben über Netzwerkschnittstellen notwendig sind},
	plural={einfache Gesten}
}

\newglossaryentry{synchronegeste}
{
	name=synchrone Geste,
	description={Synchrone Gesten bezeichnen Bewegungsmuster, die von mehreren Benutzern bzw. von einem Benutzer auf mehreren Geräten gleichzeitig ausgeführt werden. Eine synchrone Geste wird dann erkannt, wenn komplementäre Anteile der Geste zeitlich synchron oder sequentiell aufeinanderfolgend von verschiedenen Geräten beigesteuert und als zusammengehörig erkannt werden \citep{Hinckley2003}}
}

\newglossaryentry{accelerometer}
{
	name=Accelerometer,
	description={Engl. für Beschleunigungssensor. Misst und liefert kontinuierlich Daten über die Beschleunigung, also die aktuelle Geschwindigkeitsänderung eines Geräts}
}

\newglossaryentry{vermittlungskomponente}
{
	name=Vermittlungskomponente,
	description={Ein Server, der von den beteiligten Geräten die relevanten Daten einer Geste empfängt, miteinander vergleicht und entscheidet, ob eine synchrone Geste \textit{erfolgreich} oder \textit{nicht erfolgreich} ausgeführt wurde. Der Server kann sich entweder lokal auf einem der Geräte befinden oder über das Internet angesprochen werden}
}

\newglossaryentry{pinch}
{
	name=Pinch,
	description={Engl. für kneifen. Eine für Touchscreen-Geräte viel verwendete Geste, bei der der Nutzer mit Daumen und Zeigefinger den Bildschirm berührt und sie zusammenführt bis sie sich berühren. Typischerweise zum herauszoomen verwendet}
}

\newglossaryentry{approach}
{
	name=Approach,
	description={Engl. für sich nähern. Über Wireless-Technologien (z.B. Bluetooth) wird der Abstand zwischen Geräten bestimmt. Werden bestimmte Grenzwerte überschritten wird eine Annäherung als Approach erkannt}
}

\newglossaryentry{stitch}
{
	name=Stitch,
	description={Engl. für nähen. Ein Swipe, der auf dem Bildschirm eines Gerätes beginnt, über die Bildschirmgrenzen geführt wird und auf dem Bildschirm eines zweiten Geräts endet}
}
