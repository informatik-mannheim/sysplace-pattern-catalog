\documentclass[11pt,a4paper,notitlepage]{article}

\usepackage[utf8]{inputenc}
\usepackage[T1]{fontenc}
\usepackage[german]{babel}
\usepackage{float}
%\usepackage{amsmath}
%\usepackage{amsfonts}
\usepackage{amssymb}
\usepackage{graphicx}
\usepackage{ifthen}
\usepackage{diagbox}
\usepackage[hyphens]{url}
\usepackage{textcomp,gensymb}
\usepackage{makecell}
\usepackage{textpos}
\usepackage{tabularx}
\usepackage{csquotes}
%\usepackage{hyperref}
\usepackage[natbib=true,bibstyle=numeric,backend=bibtexu,citestyle=numeric]{biblatex}
\bibliography{lit.bib} 
\renewcommand\theadalign{cb}
\renewcommand\theadfont{\bfseries}
\renewcommand\theadgape{\Gape[4pt]}

\usepackage{nopageno}

\author{}
\date{}
\title{\name}

% Template für Checkboxen ----
\newcommand{\checkbox}[1]{
\ifx#1\undefined
  $\Box$
\else
  $\boxtimes$  
\fi}

\setlength{\parindent}{0pt}

%----------------------------

\newcommand{\grafischedarstellung}{\jobname_graphical_description.png}

%----------------------------

\newcommand{\umldiagram}{\jobname_uml.png}

%----------------------------

\newcommand{\sequencediagram}{\jobname_sequence.png}

%----------------------------

\newcommand{\solutionimg}{\jobname_solution.png}

%----------------------------

\newcommand{\prototypeimg}{\jobname_prototype.png}

% --------- Glossary
\newcommand{\sen}{Sender}
\newcommand{\rec}{Empfänger}
\newcommand{\recdev}{Empfangsgerät}
\newcommand{\sendev}{Sendegerät}
\newcommand{\data}{Datenobjekt}

\newcommand{\name}{Pinch To Extend}

% -------------------------------
% WAS
% -------------------------------

\newcommand{\desc}{Der Bildschirm-Inhalt des Quell-Geräts soll auf dem Ziel-Gerät erweitert dargestellt werden.}

\newcommand{\solution}{Ein Anwender platziert zwei Geräte unmittelbar nebeneinander und berührt mit je einem Finger die Displays der Geräte. Die beiden aufliegenden Finger zieht er zusammen. Dadurch wird eine Erweiterung des Bildschirminhalts des Quell-Geräts auf das Ziel-Gerät veranlasst.}

%\newcommand{\category}{give}
%\newcommand{\category}{take}
%\newcommand{\category}{exchange}
\newcommand{\category}{extend}
%\newcommand{\category}{connect}

% -------------------------------
% WIE
% -------------------------------

\newcommand{\useraction}{Der Anwender legt ein Quell-Gerät an ein Ziel-Gerät an. Dann platziert der Anwender einen Finger (z.B. Zeigefinger) auf dem Display des Quell-Geräts. Anschließend platziert der Anwender einen anderen Finger (z.B. Daumen) auf dem Display des Ziel-Geräts. Die beiden Finger zieht er zusammen bis sie sich berühren.}

\newcommand{\reaction}{Durch Auflegen des ersten Fingers wird das Quell-Gerät bestimmt. Durch Auflegen des zweiten Fingers das Ziel-Gerät. Sobald beide Finger außerhalb des Displayrandes beider Geräte gezogen wurden, wird der Bildschirm-Inhalt des Quell-Geräts auf dem Ziel-Gerät erweitert dargestellt.}

\newcommand{\reactionSuccessVisual}{}
%\newcommand{\reactionSuccessAcustic}{}
%\newcommand{\reactionSuccessSensitive}{}

\newcommand{\reactionFailureConnection}{}
\newcommand{\reactionFailureConnectionDesc}{muss dem Anwender durch einen Hinweis mitgeteilt werden, dass die Interaktion nicht erfolgreich war - der Bildschirm kann nicht erweitert auf einem weiteren Gerät dargestellt werden.}

\newcommand{\reactionFailureNoDevice}{}
\newcommand{\reactionFailureNoDeviceDesc}{muss der Anwender ggf. die Interaktion erneut ausführen. Solange keine erfolgreiche Bildschirm-Erweiterung stattgefunden hat, sollte dem Anwender signalisiert werden, dass das Zielgerät nicht erkannt wurde.}

\newcommand{\reactionFailureCompatibility}{}
\newcommand{\reactionFailureCompatibilityDesc}{sollte der Anwender vor erneuter Ausführung der Interaktion informiert werden, welche Geräte kompatibel sind und die Interaktion erneut ausführen.}

\newcommand{\designnotes}{
\begin{itemize}
\item Es sollte im besten Fall keinen Abstand zwischen den beiden Geräten geben, um die Erkennung des angelegten Geräts zu gewährleisten.
\end{itemize}}

% -------------------------------
% WANN
% -------------------------------

\newcommand{\validcontext}{Bildschirmerweiterung}

\newcommand{\simultaneously}{}
%\newcommand{\sequentially}{}

\newcommand{\online}{}
%\newcommand{\offline}{}

\newcommand{\private}{}
\newcommand{\semipublic}{}
%\newcommand{\public}{}
\newcommand{\stationary}{}
%\newcommand{\onthego}{}

%\newcommand{\leanback}{}
\newcommand{\leanforward}{}

\newcommand{\single}{}
\newcommand{\collaboration}{}
\newcommand{\facetoface}{}
\newcommand{\sidetoside}{}
\newcommand{\cornertocorner}{}

%\newcommand{\smalltask}{}
%\newcommand{\repeatedtask}{}
%\newcommand{\locationbased}{}
%\newcommand{\distraction}{}
%\newcommand{\urgent}{} 

\newcommand{\notvalidcontext}{...}


\newcommand{\devicetabular}{
\begin{tabular}[H]{|c|c|c|c|c|c|}
\hline 
\diagbox{von}{nach}   & Smartwatch & Smartphone & Tablet & Tabletop & Screens \\ 
\hline 
Smartwatch            &            &            &        &          &         \\ 
\hline 
Smartphone            &            &     x      &   x    &          &     x   \\ 
\hline 
Tablet                &            &     x      &   x    &          &     x   \\ 
\hline 
Tabletop              &            &            &        &          &         \\ 
\hline
Screens               &            &            &        &          &         \\ 
\hline 
\end{tabular} }

% -------------------------------
% WARUM
% -------------------------------

%\newcommand{\established}{}
\newcommand{\candidate}{}
%\newcommand{\realizable}{}
%\newcommand{\futuristic}{}

\newcommand{\otherpatterns}{
\begin{itemize}
\item Approach To Extend - Der Nutzer nähert sich mit dem Quell-Gerät an das Ziel-Gerät.
\item Stitch To Extend - Der Bildschirm-Inahlt des Quell-Geräts wird durch eine Bewegung mit dem Finger von dem Quell-Display auf das Ziel-Display erweitert.
\item Appose - Der Bildschirm-Inhalt des Quell-Geräts wird durch Anlegen eines Ziel-Geräts auf diesem erweitert dargestellt. 
\end{itemize}}

\newcommand{\stateoftheart}{
%\begin{enumerate}
%\item
%\item
%\end{enumerate}
}


\newcommand{\designprinciples}{}

\newcommand{\imageschemata}{}
\newcommand{\imageSchemaContainer}{}
%\newcommand{\imageSchemaInOut}{}
%\newcommand{\imageSchemaPath}{}
%\newcommand{\imageSchemaSourcePathGoal}{}
%\newcommand{\imageSchemaUpDown}{}
\newcommand{\imageSchemaLeftRight}{}
%\newcommand{\imageSchemaNearFar}{}
%\newcommand{\imageSchemaPartWhole}{}

\newcommand{\realworld}{}
\newcommand{\realworldNaivePhysic}{}
\newcommand{\realworldBodyAwareness}{}
\newcommand{\realworldEnvironmentAwareness}{}
\newcommand{\realworldSocialAwareness}{}

\newcommand{\metaphor}{}
\newcommand{\metaphordesc}{Bildschirmerweiterung, etwas herüberziehen}

% -------------------------------
% TECHNISCHES
% -------------------------------

\newcommand{\technologyObjectIntimate}{}
%\newcommand{\technologyObjectPersonal}{}
%\newcommand{\technologyObjectSocial}{}
%\newcommand{\technologyObjectPublic}{}

\newcommand{\technologyObjectDesc}{...}

%\newcommand{\technologyCommunicationServer}{}
%\newcommand{\technologyCommunicationAdhoc}{}

\newcommand{\technologyCommunicationDesc}{...}

%\newcommand{\technologyOrientationAccelerometer}{}
%\newcommand{\technologyOrientationGPS}{}
%\newcommand{\technologyOrientationGyroskop}{}
%\newcommand{\technologyOrientationAnnaeherung}{}
%\newcommand{\technologyOrientationHoehe}{}
%\newcommand{\technologyOrientationBeacons}{}
%\newcommand{\technologyOrientationOther}{}

\newcommand{\technologyOrientationDesc}{...}

\newcommand{\prototype}{...}


% -------------------------------
% SONSTIGES
% -------------------------------

\newcommand{\authors}{...}
\newcommand{\literature}{...}
\newcommand{\figures}{...}
\newcommand{\versionhistory}{...}
\newcommand{\dateofcreation}{...}
\newcommand{\comments}{...}
\newcommand{\questions}{...}


% template inkludieren --------------

\maketitle


\section*{Was}

\subsection*{Problem}
\desc

\subsection*{Lösung}
\solution

\subsection*{Grafische Darstellung}

\begin{figure}[H]
\includegraphics[scale=0.3]{mypicture.png}
\end{figure}


\subsection*{Kategorie}
\ifthenelse{\equal{\category}{give}}{$\boxtimes$}{$\Box$} Give \\
\ifthenelse{\equal{\category}{take}}{$\boxtimes$}{$\Box$} Take \\
\ifthenelse{\equal{\category}{exchange}}{$\boxtimes$}{$\Box$} Exchange \\
\ifthenelse{\equal{\category}{extend}}{$\boxtimes$}{$\Box$} Extend \\
\ifthenelse{\equal{\category}{connect}}{$\boxtimes$}{$\Box$} Connect



\section*{Wie}

\subsection*{Aktion des Benutzers}
\useraction

\subsection*{Reaktion des Sende-und Empfänger-Gerätes}
%\reaction

\subsection*{Hinweise zur Gestaltung der Interaktion}
%\designnotes



\section*{Wann}

\subsection*{Geeigneter Nutzungskontext}

\subsubsection*{Zeit}
\ifthenelse{\equal{\when}{gleichzeitig}}{$\boxtimes$}{$\Box$} gleichzeitige Nutzung von Geräten \\
\ifthenelse{\equal{\when}{aufeinanderfolgend}}{$\boxtimes$}{$\Box$} sequentielle Nutzung von Geräten 

\subsubsection*{Modus}
\ifthenelse{\equal{\mode}{online}}{$\boxtimes$}{$\Box$} online \\
\ifthenelse{\equal{\mode}{offline}}{$\boxtimes$}{$\Box$} offline \\

%\validcontext

\subsection*{Abzuratender Nutzungskontext}
%\notvalidcontext

\subsection*{Geräteklassen}
\begin{tabular}{|c|c|c|c|c|}
\hline 
• & • & Mittel & Riesig & Groß \\ 
\hline 
• & • & • & • & • \\ 
\hline 
• & • & • & • & • \\ 
\hline 
• & • & • & • & • \\ 
\hline 
• & • & • & • & • \\ 
\hline 
\end{tabular} 

\subsection*{Entfernung zwischen Sender- und Empfänger-Gerät}



\section*{Warum}


\subsection*{Displaygrößen}


\subsection*{Analoge Patterns}


\subsection*{State of the Art/Gebrauchshistorie}


\subsection*{Checkliste: Entspricht die Interaktion der Definiton eines "Blended Interaction"?}


\section*{Technisches}

\subsection*{Technologien zur Objekterkennung}


\subsection*{Technologien zur Kommunikation}


\subsection*{Technologien zur Bewegungs-/Orientierungsbestimmung}


\subsection*{Prototyp/Lösungsansatz/Code-Snippets/UML-Diagramm}



\section*{Sonstiges}

\subsection*{Autor/en}

\subsection*{Literaturreferenzen}

\subsection*{Abbildungsverzeichnis}

\subsection*{Versionshistorie}

\subsection*{Kommentare}

\subsection*{Offene Fragen}