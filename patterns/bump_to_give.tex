\documentclass[11pt,a4paper,notitlepage]{article}

\usepackage[utf8]{inputenc}
\usepackage[T1]{fontenc}
\usepackage[german]{babel}
\usepackage{float}
%\usepackage{amsmath}
%\usepackage{amsfonts}
\usepackage{amssymb}
\usepackage{graphicx}
\usepackage{ifthen}
\usepackage{diagbox}
\usepackage[hyphens]{url}
\usepackage{textcomp,gensymb}
\usepackage{makecell}
\usepackage{textpos}
\usepackage{tabularx}
\usepackage{csquotes}
%\usepackage{hyperref}
\usepackage[natbib=true,bibstyle=numeric,backend=bibtexu,citestyle=numeric]{biblatex}
\bibliography{lit.bib} 
\renewcommand\theadalign{cb}
\renewcommand\theadfont{\bfseries}
\renewcommand\theadgape{\Gape[4pt]}

\usepackage{nopageno}

\author{}
\date{}
\title{\name}

% Template für Checkboxen ----
\newcommand{\checkbox}[1]{
\ifx#1\undefined
  $\Box$
\else
  $\boxtimes$  
\fi}

\setlength{\parindent}{0pt}

%----------------------------

\newcommand{\grafischedarstellung}{\jobname_graphical_description.png}

%----------------------------

\newcommand{\umldiagram}{\jobname_uml.png}

%----------------------------

\newcommand{\sequencediagram}{\jobname_sequence.png}

%----------------------------

\newcommand{\solutionimg}{\jobname_solution.png}

%----------------------------

\newcommand{\prototypeimg}{\jobname_prototype.png}

% --------- Glossary
\newcommand{\sen}{Sender}
\newcommand{\rec}{Empfänger}
\newcommand{\recdev}{Empfangsgerät}
\newcommand{\sendev}{Sendegerät}
\newcommand{\data}{Datenobjekt}

\newcommand{\name}{Bump To Give}

% -------------------------------
% WAS
% -------------------------------

\newcommand{\desc}{Ein Benutzer (der \sen{}) hat ein \data\ und will es mit einem weiteren Benutzer (dem \rec{}) bzw. einem \recdev\ teilen. Es soll durch direkten Kontakt vom \sendev\ auf das \recdev\ übertragen werden, sodass es anschließend auf beiden Geräten verfügbar ist.}

\newcommand{\solution}{Der \sen\ hat das \sendev\ in der Hand. Durch das  zusammenstoßen (\glslink{bump}{Bumpen}) des \sendev s\ mit dem \recdev\ wird das \data\ auf das \recdev\ übertragen.}

\newcommand{\category}{give}
%\newcommand{\category}{take}
%\newcommand{\category}{exchange}
%\newcommand{\category}{extend}
%\newcommand{\category}{connect}

% -------------------------------
% WIE
% -------------------------------`

\newcommand{\useraction}{Der \sen\ hält das \sendev\ in der Hand und stößt es leicht an das \recdev\ an. Das \recdev\ kann dabei von einer weiteren Person gehalten werden oder stationär sein (z.B ein Tablet oder Tabletop). Ein \data\ ist entweder explizit vom \sen\ (z.B. ein Foto) oder implizit durch die Applikation (z.B. der
aktuelle Bildschirm) zum Versenden ausgewählt worden. Zudem besteht eine
Verbindung zum \recdev{}.}

\newcommand{\reactionSen}{Auf dem \sendev\ sollte zu verschiedenen Phasen des Bumps Feedback gegeben werden. Ein Bump besteht aus den drei \glslink{atomareinteraktion}{Atomaren Interaktionen}: \textit{Move}, \textit{Abrupt Stop} und \textit{Bump Recognized}.\\
Da es sich bei \textbf{Bump to Give} um eine \gls{synchronegeste} handelt, werden nach dem Bump die Acceleratordaten der beteiligten Geräte verglichen. Sind die Daten ähnlich genug, gilt der Bump als \textit{erfolgreich}. Bei einem \textit{erfolgreich} ausgeführten Bump finden alle Atomaren Interaktionen statt, nach Beendigung der letzten wird der Transfer des selektierten \data s\ ausgeführt. Bei \textit{nicht erfolgreichem} Bump wird kein Transfer ausgeführt.}

\newcommand{\reactionRec}{Handelt es sich bei dem \recdev\ um ein mobiles Gerät, dass von einem Nutzer gehalten wird, werden die selben Atomaren Interaktionen ausgeführt wie auf dem \sendev .\ Bei einem erfolgreichen Datentransfer (Atomare Reaktion \textit{Receive}) sollte dem \rec\ Feedback über den Empfang des \data s\ gegeben werden.}

\newcommand{\microinteractionstabular}{
\begin{figure}[H]
\begin{table}[H]
\renewcommand{\arraystretch}{2}\addtolength{\tabcolsep}{-2pt}
\centering
\newcolumntype{b}{X}
\newcolumntype{t}{>{\hsize=.3\hsize}X}
\newcolumntype{s}{>{\hsize=.2\hsize}c}
\newcolumntype{m}{>{\hsize=.6\hsize}X}
\begin{tabularx}{\textwidth}{tsbbm}
\thead[X]{Name} & \thead[c]{Typ*} & \thead[X]{Trigger} & \thead[X]{Regeln} & \thead[X]{Feedback} \\
\hline
Move & M & Gerät wurde bewegt (Acceleratordaten) & \data\ ist ausgewählt &  Animation 1 \\ 
\hline
Abrupt Stop & M & Das Gerät ist mit etwas hartem zusammengestoßen (Acceleratordaten) & Acceleratordaten erfüllen Bump-Anforderungen & Animation 2 \\ 
\hline
Bump Recognized & S & Bump-Daten von anderem Gerät empfangen und mit eigenen verglichen & Acceleratordaten beider Geräte ähnlich genug & Animation 3 \\ 
\hline
Receive & S & Daten empfangen & \data\ ist darstellbar & Animation 4 \newline Vibration \\
\hline
\end{tabularx}
\end{table}
\caption{Atomare Interaktionen für das Bump to Give Pattern}
\end{figure}
*Typ: (M)anuell, (S)ystem
}

\newcommand{\animations}{
\begin{enumerate}
\item Move-Animation: visualisiert dem Benutzer, dass das physische Bewegen des Geräts eine Funktionalität darstellt (z.B. Das Datenobjekt bewegt sich etwas verzögert, als hätte es Masse)
\item Abrupt Stop-Animation: visualisiert dem Benutzer, dass ein Bump auf seinem Gerät erkannt wurde (z.B. Das \data\ bleibt am angestoßenen Rand des Bildschirms)
\item Bump Recognized-Animation: visualisiert dem Benutzer, dass die Bump-Geste richtig erkannt wurde und das \data\ versandt wurde (z.B. Das \data\ verlässt den Bildschirm in Richtung des \recdev{}s)
\item Receive-Animation: visualisiert dem Benutzer, dass ein \data\ empfangen wurde (z.B. Das \data\ kommt auf den Bildschirm aus der Richtung des \sendev{}s)
\end{enumerate}
}

\newcommand{\designnotes}{ 
Beim Ausführen der Bump-Geste sollten die Geräte direkt aneinandergestoßen werden um eine möglichst gute Erkennung zu gewährleisten. Zudem sollte die Härte des Zusammenstoßes stark genug sein um eine Erkennung zu ermöglichen, jedoch die Geräte nicht beschädigen.}

% -------------------------------
% WANN
% -------------------------------
\newcommand{\validcontext}{Datenaustausch von Bildern, Videos, Visitenkarten, Social Network IDs, Systembefehlen}

\newcommand{\simultaneously}{}
%\newcommand{\sequentially}{}

\newcommand{\private}{}
\newcommand{\semipublic}{}
\newcommand{\public}{}
\newcommand{\stationary}{}
\newcommand{\onthego}{}

\newcommand{\leanback}{}
\newcommand{\leanforward}{}

\newcommand{\single}{}
\newcommand{\collaboration}{}
\newcommand{\facetoface}{}
\newcommand{\sidetoside}{}

\newcommand{\notvalidcontext}{--- keine Information ---}

\newcommand{\devicetabular}{
\begin{figure}[H]
\begin{tabular}{|c|c|c|c|c|c|}
\hline 
\diagbox{von}{nach} & Smartwatch & Smartphone & Tablet & Tabletop & Screens \\ 
\hline 
Smartwatch          &     x      &     x      &   x    &     x    &        \\ 
\hline 
Smartphone          &     x      &     x      &   x    &     x    &        \\ 
\hline 
Tablet              &     x      &     x      &   x    &     x    &        \\ 
\hline 
Tabletop            &            &            &        &          &        \\ 
\hline
Screens             &            &            &        &          &         \\ 
\hline 
\end{tabular}
\caption{Geräteklassen für das Bump To Give Pattern} 
\end{figure}
}

% -------------------------------
% WARUM
% -------------------------------

%\newcommand{\established}{}
\newcommand{\candidate}{}
\newcommand{\realizable}{}
%\newcommand{\futuristic}{}

\newcommand{\otherpatterns}{
\begin{itemize}
\item Bump To Take
\item Bump To Exchange
\item Bump To Connect
\item Nudge
\end{itemize}
}

\newcommand{\stateoftheart}{
\begin{enumerate}
\item Bump App: Bis 2014 in den App/Play Stores erhältlich gewesen [\url{http://bu.mp/}]
\item Hinckley, K., Bumping Objects Together as a Semantically Rich Way of Forming Connections between Ubiquitous Devices. UbiComp 2003 Formal Video Program, Seattle, WA, Oct 12-15, 2003.
\end{enumerate}
}

\newcommand{\designprinciples}{}

\newcommand{\imageschemata}{}
%\newcommand{\imageSchemaVoid}{}
\newcommand{\imageSchemaObject}{}
%\newcommand{\imageSchemaSubstance}{}
%\newcommand{\imageSchemaCenterPeriphery}{}
\newcommand{\imageSchemaContact}{}
%\newcommand{\imageSchemaFrontBack}{}
%\newcommand{\imageSchemaLocation}{}
%\newcommand{\imageSchemaNearFar}{}
%\newcommand{\imageSchemaPath}{}
\newcommand{\imageSchemaSourcePathGoal}{}
%\newcommand{\imageSchemaScale}{}
%\newcommand{\imageSchemaLeftRight}{}
\newcommand{\imageSchemaContainer}{}
\newcommand{\imageSchemaContent}{}
%\newcommand{\imageSchemaFullEmpty}{}
\newcommand{\imageSchemaInOut}{}
%\newcommand{\imageSchemaSurface}{}
%\newcommand{\imageSchemaMerging}{}
%\newcommand{\imageSchemaSplitting}{}
\newcommand{\imageSchemaMomentum}{}
%\newcommand{\imageSchemaSelfMotion}{}
%\newcommand{\imageSchemaBigSmall}{}
\newcommand{\imageSchemaFastSlow}{}
%\newcommand{\imageSchemaPartWhole}{}

\newcommand{\realworld}{}
\newcommand{\realworldNaivePhysic}{}
\newcommand{\realworldBodyAwareness}{}
\newcommand{\realworldEnvironmentAwareness}{}
\newcommand{\realworldSocialAwareness}{}

\newcommand{\metaphor}{}
\newcommand{\metaphordesc}{Fistbump, Anstoßen von Getränken}

% -------------------------------
% TECHNISCHES
% -------------------------------

\newcommand{\technologyObjectIntimate}{}
\newcommand{\technologyObjectPersonal}{}
%\newcommand{\technologyObjectSocial}{}
%\newcommand{\technologyObjectPublic}{}

\newcommand{\technologyObjectDesc}{Über die Major- und Minor-Werte kann mit iBeacon jedem Endgerät eine einzigartige ID zugeordnet werden mit der Endgeräte identifiziert werden können. Um diese ID zwischen Bump-Partnern auszutauschen, wird auf beiden Geräten iBeacon zum Zeitpunk des Bumps für kurze Zeit aktiviert. Gerade lange genug, damit die Endgeräte alle aktiven Beacons in ihrem Umfeld sehen können. Dadurch besitzt jedes Gerät eine Liste an Beacons die zu einen bestimmten Zeitpunkt an einem Bump, in ihrer Empfangsreichweite, beteiligt waren. Können die Endgeräte jeweils nur ein anderes Beacon sehen, haben Sie ihren Bump-Partner identifiziert. Ist mehr als ein Beacon sichtbar, fanden mehrere Bumps zeitgleich statt. In diesen Fällen können die Partner, über die Entfernung der Geräte zueinander, ermittelt werden. Bei den Geräten mit dem geringsten Abstand handelt es sich um die Bump-Partner. Die Erkennung der Partnergeräte über die Entfernung macht es erforderlich, dass zeitgleiche Bumps mindestens einige Zentimeter voneinander entfernt stattfinden. Dies stellt sicher, dass eine falsche Zuordnung durch ungenaue Abstandsmessungen vermieden wird.}

%\newcommand{\technologyCommunicationServer}{}
\newcommand{\technologyCommunicationAdhoc}{}

\newcommand{\technologyCommunicationDesc}{Wird auf einem Gerät ein Bump registriert, ist der erste Schritt die Generierung von Zufallszahlen für den Major- und Minor-Wert von iBeacon. Diese Zahlen bilden eine eindeutige ID, mit der sich jedes Gerät im Netzwerk identifizieren kann. Anschließend wird iBeacon aktiviert, die Geräte können sich gegenseitig sehen, GeräteIDs lesen und die Distanz zu allen sichtbaren Beacons erfassen. Ist mehr als ein Beacon sichtbar, wird jenes ermittelt, welches die geringste Distanz zum suchenden Gerät aufweist. Die GeräteID dieses Geräts wird lokal gespeichert und iBeacon wird deaktiviert. Anschließend startet die DiscoveryPhase des Multipeer-Connectivity-Frameworks.}

\newcommand{\technologyOrientationAccelerometer}{}
%\newcommand{\technologyOrientationGPS}{}
%\newcommand{\technologyOrientationGyroskop}{}
%\newcommand{\technologyOrientationAnnaeherung}{}
%\newcommand{\technologyOrientationHoehe}{}
\newcommand{\technologyOrientationBeacons}{}
%\newcommand{\technologyOrientationOther}{}

\newcommand{\technologyOrientationDesc}{-}

\newcommand{\prototype}{
Die Bump-Interaktion besteht aus drei Teilsystemen:
Eines dieser Systeme muss erkennen, wenn Endgeräte angestoßen werden. Ein weiteres System muss Endgeräte identifizieren können, damit die Daten zwischen den
richtigen Geräten ausgetauscht werden. Außerdem wird ein System benötigt, über das ein Kommunikationskanal zwischen den Geräten hergestellt wird, über den Daten
ausgetauscht werden können.\\

Es wurden drei iOS Frameworks genutzt für die Umsetzung von Bump:
\begin{enumerate}
\item Core Motion Framework: bietet Applikationen Zugang zu Sensordaten (Beschleunigungssensor) die durch die Gerätehardware erfasst werden
\item Core Location Framework: bietet Zugriff auf die Technologien GPS und iBeacon, mit denen standortbasierte Dienste realisiert werden können.
\item Multipeer Connectivity Framework: bietet Kommunikation für iOS und MacOS Geräte über WLAN oder Bluetooth zu kommunizieren.
\end{enumerate}

Wichtig ist, dass sich die Geräte gegenseitig anstoßen, um den Verbindungsaufbau und den Datenaustausch zwischen den Geräten auszulösen. Der dabei auftretende Zusammenprall der Geräte aneinander erzeugt Kräfte, die durch Beschleunigungssensoren gemessen werden können. Auf
dieser Grundlage wurde ein Algorithmus entwickelt, der die durch den Aufprall erzeugten Daten des Accelerometers auswertet, um zu erkennen, wann ein Bump stattgefunden hat.\\
}


% -------------------------------
% SONSTIGES
% -------------------------------

\newcommand{\authors}{
Benjamin Grab, Hochschule Mannheim\\
Valentina Burjan, Hochschule Mannheim}
\newcommand{\literature}{
\begin{enumerate}
\item Bump. [Online]. \url{http://bu.mp/}
\item BumpTechnologies. Youtube.com. [Online]. \url{https://www.youtube.com/user/BumpTechnologies}
\item Ken Hinckley. 2003. Synchronous gestures for multiple persons and computers. In Proceedings of the 16th annual ACM symposium on User interface software and technology (UIST '03). ACM, New York, NY, USA, 149-158. DOI=10.1145/964696.964713 
\item \url{https://github.com/informatik-mannheim/thesis-bump-to-transfer/tree/master/sources/
Bumper}\\
Swift-Code zu der Demonstrator Applikationen von Bump
\end{enumerate}}
\newcommand{\figures}{...}
\newcommand{\versionhistory}{...}
\newcommand{\dateofcreation}{17. September 2015}
\newcommand{\comments}{...}
\newcommand{\questions}{...}

% template inkludieren --------------

\maketitle


\section*{Was}

\subsection*{Problem}
\desc

\subsection*{Lösung}
\solution

\subsection*{Grafische Darstellung}

\begin{figure}[H]
\includegraphics[scale=0.3]{mypicture.png}
\end{figure}


\subsection*{Kategorie}
\ifthenelse{\equal{\category}{give}}{$\boxtimes$}{$\Box$} Give \\
\ifthenelse{\equal{\category}{take}}{$\boxtimes$}{$\Box$} Take \\
\ifthenelse{\equal{\category}{exchange}}{$\boxtimes$}{$\Box$} Exchange \\
\ifthenelse{\equal{\category}{extend}}{$\boxtimes$}{$\Box$} Extend \\
\ifthenelse{\equal{\category}{connect}}{$\boxtimes$}{$\Box$} Connect



\section*{Wie}

\subsection*{Aktion des Benutzers}
\useraction

\subsection*{Reaktion des Sende-und Empfänger-Gerätes}
%\reaction

\subsection*{Hinweise zur Gestaltung der Interaktion}
%\designnotes



\section*{Wann}

\subsection*{Geeigneter Nutzungskontext}

\subsubsection*{Zeit}
\ifthenelse{\equal{\when}{gleichzeitig}}{$\boxtimes$}{$\Box$} gleichzeitige Nutzung von Geräten \\
\ifthenelse{\equal{\when}{aufeinanderfolgend}}{$\boxtimes$}{$\Box$} sequentielle Nutzung von Geräten 

\subsubsection*{Modus}
\ifthenelse{\equal{\mode}{online}}{$\boxtimes$}{$\Box$} online \\
\ifthenelse{\equal{\mode}{offline}}{$\boxtimes$}{$\Box$} offline \\

%\validcontext

\subsection*{Abzuratender Nutzungskontext}
%\notvalidcontext

\subsection*{Geräteklassen}
\begin{tabular}{|c|c|c|c|c|}
\hline 
• & • & Mittel & Riesig & Groß \\ 
\hline 
• & • & • & • & • \\ 
\hline 
• & • & • & • & • \\ 
\hline 
• & • & • & • & • \\ 
\hline 
• & • & • & • & • \\ 
\hline 
\end{tabular} 

\subsection*{Entfernung zwischen Sender- und Empfänger-Gerät}



\section*{Warum}


\subsection*{Displaygrößen}


\subsection*{Analoge Patterns}


\subsection*{State of the Art/Gebrauchshistorie}


\subsection*{Checkliste: Entspricht die Interaktion der Definiton eines "Blended Interaction"?}


\section*{Technisches}

\subsection*{Technologien zur Objekterkennung}


\subsection*{Technologien zur Kommunikation}


\subsection*{Technologien zur Bewegungs-/Orientierungsbestimmung}


\subsection*{Prototyp/Lösungsansatz/Code-Snippets/UML-Diagramm}



\section*{Sonstiges}

\subsection*{Autor/en}

\subsection*{Literaturreferenzen}

\subsection*{Abbildungsverzeichnis}

\subsection*{Versionshistorie}

\subsection*{Kommentare}

\subsection*{Offene Fragen}