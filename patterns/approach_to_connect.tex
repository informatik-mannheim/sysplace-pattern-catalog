\documentclass[11pt,a4paper,notitlepage]{article}

\usepackage[utf8]{inputenc}
\usepackage[T1]{fontenc}
\usepackage[german]{babel}
\usepackage{float}
%\usepackage{amsmath}
%\usepackage{amsfonts}
\usepackage{amssymb}
\usepackage{graphicx}
\usepackage{ifthen}
\usepackage{diagbox}
\usepackage[hyphens]{url}
\usepackage{textcomp,gensymb}
\usepackage{makecell}
\usepackage{textpos}
\usepackage{tabularx}
\usepackage{csquotes}
%\usepackage{hyperref}
\usepackage[natbib=true,bibstyle=numeric,backend=bibtexu,citestyle=numeric]{biblatex}
\bibliography{lit.bib} 
\renewcommand\theadalign{cb}
\renewcommand\theadfont{\bfseries}
\renewcommand\theadgape{\Gape[4pt]}

\usepackage{nopageno}

\author{}
\date{}
\title{\name}

% Template für Checkboxen ----
\newcommand{\checkbox}[1]{
\ifx#1\undefined
  $\Box$
\else
  $\boxtimes$  
\fi}

\setlength{\parindent}{0pt}

%----------------------------

\newcommand{\grafischedarstellung}{\jobname_graphical_description.png}

%----------------------------

\newcommand{\umldiagram}{\jobname_uml.png}

%----------------------------

\newcommand{\sequencediagram}{\jobname_sequence.png}

%----------------------------

\newcommand{\solutionimg}{\jobname_solution.png}

%----------------------------

\newcommand{\prototypeimg}{\jobname_prototype.png}

% --------- Glossary
\newcommand{\sen}{Sender}
\newcommand{\rec}{Empfänger}
\newcommand{\recdev}{Empfangsgerät}
\newcommand{\sendev}{Sendegerät}
\newcommand{\data}{Datenobjekt}

\newcommand{\name}{Approach To Connect}

% -------------------------------
% WAS
% -------------------------------

\newcommand{\desc}{Ein Benutzer möchte zur Datenübertragung eine Verbindung zwischen seinem Gerät und einem weiteren, in der Nähe befindlichen Gerät aufbauen.}

\newcommand{\solution}{Ein Benutzer nähert (\textit{approach}) sich mit seinem Gerät (dem \sendev{}) einem entfernten Gerät (dem \recdev{}). Nähert er sich dem \recdev{} bis auf einen bestimmten räumlichen Abstand, wird eine Verbindung hergestellt, die zur Datenübertragung genutzt werden kann.}

%\newcommand{\category}{give}
%\newcommand{\category}{take}
%\newcommand{\category}{exchange}
%\newcommand{\category}{extend}
\newcommand{\category}{connect}

% -------------------------------
% WIE
% -------------------------------

\newcommand{\useraction}{Der Benutzer hält ein mobiles \sendev{} in der Hand und nähert sich einem \recdev{} (z.B. einem großen Wanddisplay oder einem Tablet auf dem Tisch). Zwischen \sendev{} und \recdev{} besteht keine Verbindung. Wird nun ein vordefinierter Mindestabstand (Abstands-Schwellwert) unterschritten, initiiert eines der Geräte (üblicherweise das \sendev{}) den Verbindungsaufbau, indem es einen Verbindungsanfrage an das andere sendet.}

\newcommand{\reactionSen}{Besteht keine Verbindung zwischen den Geräten, ist das \sendev{} in einer Art Monitoring-Modus, in dem es auf Geräte in der Umgebung wartet, was dem Benutzer visuell durch statisches oder kontinuierliches visuelles Feedback sichtbar gemacht werden kann. Unterschreitet der Benutzer den Abstands-Schwellwert, erhält er auf dem \sendev{} ein Feedback über das erkannte \recdev{} und den Beginn des Verbindungsaufbaus (z.B. akustisch oder visuell). Bei \textit{erfolgreichem} oder \textit{gescheiterten} Verbindungsaufbau erhält der Benutzer ein entsprechendes Feedback, das wiederum z.B. aus einem akustischen Signal oder Vibration bestehen kann.}

\newcommand{\reactionRec}{Das Empfangsgerät hat eine eher passive Rolle und sollte dann Feedback geben, wenn ein mobiles Gerät erkannt wurde und ein Verbindungsaufbau stattfindet, sodass die Reaktion des \sendev{}s und des \recdev{}s gleichzeitig stattfinden und die Zuordnung der Geräte zueinander über das Feedback sichtbar wird.}

\newcommand{\microinteractionstabular}{\textbf{Dominick}}
\newcommand{\animations}{\textbf{Dominick}}

\newcommand{\designnotes}{Der Abstands-Schwellwert darf nicht zu niedrig definiert werden, da  man sonst das \sendev{} auf das \recdev{} auflegen müsste bzw. sehr nah vor einem großen Bildschirm stehen würde. Zudem sollte die Annäherung eindeutig sein, d.h. es sollten innerhalb eines Interaktionskontexts keine konkurrierenden Annäherungsgesten möglich sein.}

% -------------------------------
% WANN
% -------------------------------

\newcommand{\validcontext}{Verbinden von mobilen und stationären Geräten, Verbinden von Privatgeräten mit (halb-)öffentlichen Displays.}

\newcommand{\simultaneously}{}
%\newcommand{\sequentially}{}

\newcommand{\online}{}
\newcommand{\offline}{}

\newcommand{\private}{}
\newcommand{\semipublic}{}
%\newcommand{\public}{}
\newcommand{\stationary}{}
%\newcommand{\onthego}{}

%\newcommand{\leanback}{}
\newcommand{\leanforward}{}

\newcommand{\single}{}
\newcommand{\collaboration}{}
\newcommand{\facetoface}{}
%\newcommand{\sidetoside}{}
%\newcommand{\cornertocorner}{}

%\newcommand{\smalltask}{}
%\newcommand{\repeatedtask}{}
%\newcommand{\locationbased}{}
%\newcommand{\distraction}{}
%\newcommand{\urgent}{} 

\newcommand{\notvalidcontext}{Aufbauen einer Verbindung zu vertraulichen Geräten oder zum Austausch vertraulicher Daten; Sichtbarmachen vertraulicher Informationen (z.B. Name, Alter etc.) auf öffentlichen Displays.}


\newcommand{\devicetabular}{
\begin{tabular}[H]{|c|c|c|c|c|c|}
\hline 
\diagbox{von}{nach}   & Smartwatch & Smartphone & Tablet & Tabletop & Screens \\ 
\hline 
Smartwatch            &            &     x      &   x    &     x    &     x   \\ 
\hline 
Smartphone            &            &     x      &   x    &     x    &     x   \\ 
\hline 
Tablet                &            &            &   x    &     x    &     x   \\ 
\hline 
Tabletop              &            &            &        &          &         \\ 
\hline
Screens               &            &            &        &          &         \\ 
\hline 
\end{tabular} }

% -------------------------------
% WARUM
% -------------------------------

%\newcommand{\established}{}
\newcommand{\candidate}{}
\newcommand{\realizable}{}
%\newcommand{\futuristic}{}

\newcommand{\otherpatterns}{
\begin{itemize}
\item Approach To Give
\item Approach To Take
\item Approach To Extend
\item Leave To Disconnect
\end{itemize}
}

\newcommand{\stateoftheart}{
\begin{enumerate}
\item Aufbauen einer Bluetooth-Verbindung beim Betreten des Raums \cite{Dachselt2009}.
\item Kommunikation verschiedener Personen mit einem Fernseher basierend auf der Nähe \cite{Greenberg2011}.
\item Theoretische und praktische Ansätze zu \textit{Proxemics}, behandelt auch negative Nutzungskontexte (sog. Dark Patterns) \cite{Marquardt2015}.
\end{enumerate}
}


\newcommand{\designprinciples}{}

\newcommand{\imageschemata}{}
%\newcommand{\imageSchemaVoid}{}
%\newcommand{\imageSchemaObject}{}
%\newcommand{\imageSchemaSubstance}{}
\newcommand{\imageSchemaCenterPeriphery}{}
%\newcommand{\imageSchemaContact}{}
%\newcommand{\imageSchemaFrontBack}{}
%\newcommand{\imageSchemaLocation}{}
\newcommand{\imageSchemaNearFar}{}
\newcommand{\imageSchemaPath}{}
%\newcommand{\imageSchemaSourcePathGoal}{}
%\newcommand{\imageSchemaScale}{}
%\newcommand{\imageSchemaLeftRight}{}
%\newcommand{\imageSchemaContainer}{}
%\newcommand{\imageSchemaContent}{}
%\newcommand{\imageSchemaFullEmpty}{}
\newcommand{\imageSchemaInOut}{}
%\newcommand{\imageSchemaSurface}{}
%\newcommand{\imageSchemaMerging}{}
%\newcommand{\imageSchemaSplitting}{}
%\newcommand{\imageSchemaMomentum}{}
%\newcommand{\imageSchemaSelfMotion}{}
%\newcommand{\imageSchemaBigSmall}{}
%\newcommand{\imageSchemaFastSlow}{}
%\newcommand{\imageSchemaPartWhole}{}

\newcommand{\realworld}{}
\newcommand{\realworldNaivePhysic}{}
\newcommand{\realworldBodyAwareness}{}
\newcommand{\realworldEnvironmentAwareness}{}
\newcommand{\realworldSocialAwareness}{}

\newcommand{\metaphor}{}
\newcommand{\metaphordesc}{Magnete (die sich bei geringer Distanz verbinden durch ihre Pole)}

% -------------------------------
% TECHNISCHES
% -------------------------------

\newcommand{\requiredTechnologies}{
Um Approach To Connect auf einem Gerät (\textit{Device}) ausführen zu können, gibt es einige Voraussetzungen und Einschränkungen bezüglich der verfügbaren Technologien auf diesem Gerät. Ein Gerät ist dann für das Approach To Connect Pattern verwendbar, wenn es folgende Eigenschaften aufweist:
\begin{itemize}
\item \textbf{Input}:
\item \textbf{Output}:
\item \textbf{Connectivity}:
\end{itemize}
}
\newcommand{\implementation}{
\subsubsection*{Ablauf Gestenerkennung}
Bei der Geste Approach To Connect handelt es sich um eine \textit{einfache Geste}, deren Erkennung nur auf einem Gerät durchgeführt werden muss. Der allgemeine Ablauf entfällt in zwei Teile (s. Abbildung \ref{gesture_detection}):
\begin{itemize}
\item Erkennen der Geste (\textit{Gesture Detection}) und
\item Überprüfen eventueller Bedingungen an die Geste (\textit{Constraint Check}).
\end{itemize}

\begin{figure}[h]
\includegraphics[width=\textwidth]{gesture_detection.png}
\caption{Allgemeiner Ablauf einer Gestenerkennung}
\label{gesture_detection}
\end{figure}
Wurde die Geste erkannt, wird ein entsprechendes \textit{GestureEvent} generiert, das an den \textit{Constraint Check} übergeben wird.

\subsubsection*{Approach Erkennung}
\begin{figure}[h]
\includegraphics[width=\textwidth]{approach_recognize.png}
\caption{Erkennung der Approach Geste}
\label{recognize_swipe}
\end{figure}

\subsubsection*{Approach Constraint Check}
\begin{figure}[h]
\includegraphics[width=\textwidth]{approach_check_constraints.png}
\caption{Überprüfung der Constraints für die Approach Geste}
\label{check_constraints}
\end{figure}

}

% -------------------------------
% SONSTIGES
% -------------------------------

\newcommand{\authors}{Horst Schneider, Hochschule Mannheim\\
Dominick Madden, Hochschule Mannheim\\
Valentina Burjan, Hochschule Mannheim}
\newcommand{\literature}{...}
\newcommand{\figures}{...}
\newcommand{\versionhistory}{11.08.2016}
\newcommand{\dateofcreation}{15.08.2015}
\newcommand{\comments}{...}
\newcommand{\questions}{...}


% template inkludieren --------------

\maketitle


\section*{Was}

\subsection*{Problem}
\desc

\subsection*{Lösung}
\solution

\subsection*{Grafische Darstellung}

\begin{figure}[H]
\includegraphics[scale=0.3]{mypicture.png}
\end{figure}


\subsection*{Kategorie}
\ifthenelse{\equal{\category}{give}}{$\boxtimes$}{$\Box$} Give \\
\ifthenelse{\equal{\category}{take}}{$\boxtimes$}{$\Box$} Take \\
\ifthenelse{\equal{\category}{exchange}}{$\boxtimes$}{$\Box$} Exchange \\
\ifthenelse{\equal{\category}{extend}}{$\boxtimes$}{$\Box$} Extend \\
\ifthenelse{\equal{\category}{connect}}{$\boxtimes$}{$\Box$} Connect



\section*{Wie}

\subsection*{Aktion des Benutzers}
\useraction

\subsection*{Reaktion des Sende-und Empfänger-Gerätes}
%\reaction

\subsection*{Hinweise zur Gestaltung der Interaktion}
%\designnotes



\section*{Wann}

\subsection*{Geeigneter Nutzungskontext}

\subsubsection*{Zeit}
\ifthenelse{\equal{\when}{gleichzeitig}}{$\boxtimes$}{$\Box$} gleichzeitige Nutzung von Geräten \\
\ifthenelse{\equal{\when}{aufeinanderfolgend}}{$\boxtimes$}{$\Box$} sequentielle Nutzung von Geräten 

\subsubsection*{Modus}
\ifthenelse{\equal{\mode}{online}}{$\boxtimes$}{$\Box$} online \\
\ifthenelse{\equal{\mode}{offline}}{$\boxtimes$}{$\Box$} offline \\

%\validcontext

\subsection*{Abzuratender Nutzungskontext}
%\notvalidcontext

\subsection*{Geräteklassen}
\begin{tabular}{|c|c|c|c|c|}
\hline 
• & • & Mittel & Riesig & Groß \\ 
\hline 
• & • & • & • & • \\ 
\hline 
• & • & • & • & • \\ 
\hline 
• & • & • & • & • \\ 
\hline 
• & • & • & • & • \\ 
\hline 
\end{tabular} 

\subsection*{Entfernung zwischen Sender- und Empfänger-Gerät}



\section*{Warum}


\subsection*{Displaygrößen}


\subsection*{Analoge Patterns}


\subsection*{State of the Art/Gebrauchshistorie}


\subsection*{Checkliste: Entspricht die Interaktion der Definiton eines "Blended Interaction"?}


\section*{Technisches}

\subsection*{Technologien zur Objekterkennung}


\subsection*{Technologien zur Kommunikation}


\subsection*{Technologien zur Bewegungs-/Orientierungsbestimmung}


\subsection*{Prototyp/Lösungsansatz/Code-Snippets/UML-Diagramm}



\section*{Sonstiges}

\subsection*{Autor/en}

\subsection*{Literaturreferenzen}

\subsection*{Abbildungsverzeichnis}

\subsection*{Versionshistorie}

\subsection*{Kommentare}

\subsection*{Offene Fragen}