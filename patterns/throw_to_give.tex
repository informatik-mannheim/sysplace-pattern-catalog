\documentclass[11pt,a4paper,notitlepage]{article}

\usepackage[utf8]{inputenc}
\usepackage[T1]{fontenc}
\usepackage[german]{babel}
\usepackage{float}
%\usepackage{amsmath}
%\usepackage{amsfonts}
\usepackage{amssymb}
\usepackage{graphicx}
\usepackage{ifthen}
\usepackage{diagbox}
\usepackage[hyphens]{url}
\usepackage{textcomp,gensymb}
\usepackage{makecell}
\usepackage{textpos}
\usepackage{tabularx}
\usepackage{csquotes}
%\usepackage{hyperref}
\usepackage[natbib=true,bibstyle=numeric,backend=bibtexu,citestyle=numeric]{biblatex}
\bibliography{lit.bib} 
\renewcommand\theadalign{cb}
\renewcommand\theadfont{\bfseries}
\renewcommand\theadgape{\Gape[4pt]}

\usepackage{nopageno}

\author{}
\date{}
\title{\name}

% Template für Checkboxen ----
\newcommand{\checkbox}[1]{
\ifx#1\undefined
  $\Box$
\else
  $\boxtimes$  
\fi}

\setlength{\parindent}{0pt}

%----------------------------

\newcommand{\grafischedarstellung}{\jobname_graphical_description.png}

%----------------------------

\newcommand{\umldiagram}{\jobname_uml.png}

%----------------------------

\newcommand{\sequencediagram}{\jobname_sequence.png}

%----------------------------

\newcommand{\solutionimg}{\jobname_solution.png}

%----------------------------

\newcommand{\prototypeimg}{\jobname_prototype.png}

% --------- Glossary
\newcommand{\sen}{Sender}
\newcommand{\rec}{Empfänger}
\newcommand{\recdev}{Empfangsgerät}
\newcommand{\sendev}{Sendegerät}
\newcommand{\data}{Datenobjekt}

\newcommand{\name}{Throw To Give}

% -------------------------------
% WAS
% -------------------------------

\newcommand{\desc}{Ein Benutzer (der \sen) hat ein \data\ und will es mit einem weiteren Benutzer (dem \rec) bzw. einem \recdev\ teilen. Es soll über Distanz vom \sendev\ auf das \recdev\ übertragen werden, sodass es anschließend auf beiden Geräten verfügbar ist.}

\newcommand{\solution}{Der \sen\ hat das \sendev{} in der Hand. Durch eine schwungvolle Bewegung in Richtung des \recdev{}s wird das \data{} auf das \recdev{} übertragen.}

\newcommand{\category}{give}
%\newcommand{\category}{take}
%\newcommand{\category}{exchange}
%\newcommand{\category}{extend}
%\newcommand{\category}{connect}
% -------------------------------
% WIE
% -------------------------------

\newcommand{\useraction}{Der \sen\ hält das \sendev\ in der Hand. Ein \data\ ist entweder explizit vom \sen\ (z.B. ein Foto) oder implizit durch die Applikation (z.B. der aktuelle Bildschirm) zum Versenden ausgewählt worden. \recdev\ und \sendev\ sind bereits zum bidirektionalen Datenaustausch verbunden.\\

Der \sen\ schwingt das \sendev\ von seinem aus Körper in Richtung des \recdev{}s. Die Dynamik der Throw Geste sollte dem Sender das Gefühl geben, die physikalische Distanz zum \recdev\ durch die Bewegung des Geräts zu überbrücken.}

\newcommand{\reactionSen}{Auf dem \sendev\ kann dem Nutzer zu verschiedenen Phasen des Throws Feedback gegeben werden. Ein Throw besteht aus den zwei \glslink{atomareinteraktion}{atomaren Interaktionen} \textit{Accelerate} und \textit{Decelerate}. Entspricht die Kombination der beiden atomaren Interaktionen den Kriterien eines Throws, wird das \data\ übertragen.}

\newcommand{\reactionRec}{Auf dem \recdev\ kann dem Nutzer bei erfolgreichem Throw auf dem \sendev\ und anschließendem erfolgreichen Transfer ein Feedback über den Empfang des \data s\ gegeben werden (Atomare Interaktion \textit{Receive}).}

\newcommand{\microinteractionstabular}{
\begin{figure}[H]
\begin{table}[H]
\renewcommand{\arraystretch}{2}\addtolength{\tabcolsep}{-2pt}
\centering
\newcolumntype{b}{X}
\newcolumntype{t}{>{\hsize=.3\hsize}X}
\newcolumntype{s}{>{\hsize=.2\hsize}c}
\newcolumntype{m}{>{\hsize=.6\hsize}X}
\begin{tabularx}{\textwidth}{tsbbm}
\thead[X]{Name} & \thead[c]{Typ*} & \thead[X]{Trigger} & \thead[X]{Regeln} & \thead[X]{Feedback} \\
\hline
Accelerate & M & Accelerometer des \sendev{}s erkennt Beschleunigung & Die Beschleunigungsrichtung zeigt in Richtung des \recdev{}s, \newline Beschleunigung hoch genug &  Animation 1 \\ 
\hline
Decelerate & M & Accelerometer des \sendev{}s erkennt Entschleunigung & Entschleunigung abrupt genug,\newline Die Bewegung des \sendev{}s endet in Richtung des \recdev{} & Animation 2 \newline Vibration\\ 
\hline
Receive & S & Daten empfangen & \data\ ist darstellbar & Animation 3 \newline Vibration \\
\hline
\end{tabularx}
\end{table}
\caption{Atomare Interaktionen für das Throw To Give Pattern}
\end{figure}
*Typ: (M)anuell, (S)ystem
}

\newcommand{\animations}{
\begin{enumerate}
\item Accelerate-Animation: visualisiert dem Benutzer, dass das \data\ auf Beschleunigung reagiert (z.B. durch leicht verzögerte Bewegung in Beschleunigungsrichtung).
\item Decelerate-Animation: visualisiert dem Benutzer, wenn die Kombination aus Be- und Entschleunigung nicht ausreichend dynamisch war (z.B. Das \data\ wird erst verschickt wenn es durch die Animation 1 den Bildschirm verlässt). Die Vibration bestätigt zudem, dass die Geste korrekt ausgeführt wurde.
\item Receive-Animation: visualisiert dem Benutzer, dass ein \data\ empfangen wurde (z.B. Das \data\ fliegt aus Richtung des \sendev{}s auf den Bildschirm).
\end{enumerate}
}


\newcommand{\designnotes}{
\begin{itemize}
\item Die Schwenk-Bewegung muss eine definierte Geschwindigkeit erreichen bis das System die Bewegung als Interaktion erkennt, damit der Anwender ein natürliches Empfinden hat bei der Ausführung der Interaktion.
\item Das \sendev\ sollte handlich sein, um die definierte Mindestgeschwindigkeit der Geste zu erreichen ohne dabei das Gerät zu beschädigen.
\end{itemize}}

% -------------------------------
% WANN
% -------------------------------

\newcommand{\validcontext}{Geeignet für Datenaustausch: Bilder, Videos, Dateien, Social Network IDs, Kontaktinformationen}

%\newcommand{\simultaneously}{}
\newcommand{\sequentially}{}

\newcommand{\private}{}
\newcommand{\semipublic}{}
\newcommand{\public}{}
\newcommand{\stationary}{}
\newcommand{\onthego}{}

\newcommand{\leanback}{}
%newcommand{\leanforward}{}

\newcommand{\single}{}
\newcommand{\collaboration}{}
\newcommand{\facetoface}{}
%\newcommand{\sidetoside}{}
%\newcommand{\cornertocorner}{}

%\newcommand{\smalltask}{}
%\newcommand{\repeatedtask}{}
%\newcommand{\locationbased}{}
%\newcommand{\distraction}{}
%\newcommand{\urgent}{} 

\newcommand{\notvalidcontext}{- Kritische Daten sollten nicht an (halb-) öffentliche Displays gesendet werden. \\ - Wenn viele drahtlos verbundene Geräte räumlich nah beieinander sind, ist es schwierig zu entscheiden welches der \rec\ ist. \\ - Der Nutzer sollte nicht zu nah am \recdev\ stehen, da es zu Schäden kommen könnte. Außerdem ist die Wurfmetapher erst ab einer bestimmten Entfernung gegeben.}


\newcommand{\devicetabular}{
\begin{tabular}[H]{|c|c|c|c|c|c|}
\hline 
\diagbox{von}{nach}   & Smartwatch & Smartphone & Tablet & Tabletop & Screens \\ 
\hline 
Smartwatch            &            &            &       &         &        \\ 
\hline 
Smartphone            &     x       &      x      &   x    &     x    &     x   \\ 
\hline 
Tablet                &            &            &        &          &         \\ 
\hline 
Tabletop              &            &            &        &          &         \\ 
\hline
Screens               &            &            &        &          &         \\ 
\hline 
\end{tabular} }

% -------------------------------
% WARUM
% -------------------------------

%\newcommand{\established}{}
\newcommand{\candidate}{}
\newcommand{\realizable}{}
%\newcommand{\futuristic}{}

\newcommand{\otherpatterns}{Shake to Connect}


\newcommand{\stateoftheart}{
\begin{enumerate}
\item Studienarbeit zu Interaktionen und Animationen im Multiscreen Kontext
\cite{Madden2016}.
\item Bachelorthesis zur Implementierung einer gestenbasierten Interaktion zum Datenaustausch zwischen mobilen Endgeräten \cite{Grab2015}.
\item Wissenschaftliche Erwähnung der Throw Gesten \cite{Dachselt2009}.
\end{enumerate}
}


\newcommand{\designprinciples}{}

\newcommand{\imageschemata}{}
%\newcommand{\imageSchemaCenterPeriphery}{}
%\newcommand{\imageSchemaContact}{}
%\newcommand{\imageSchemaFrontBack}{}
%\newcommand{\imageSchemaLocation}{}
\newcommand{\imageSchemaNearFar}{}
\newcommand{\imageSchemaObject}{}
\newcommand{\imageSchemaPath}{}
\newcommand{\imageSchemaSourcePathGoal}{}
%\newcommand{\imageSchemaScale}{}
%\newcommand{\imageSchemaLeftRight}{}
\newcommand{\imageSchemaContainer}{}
\newcommand{\imageSchemaContent}{}
%\newcommand{\imageSchemaFullEmpty}{}
\newcommand{\imageSchemaInOut}{}
%\newcommand{\imageSchemaSurface}{}
%\newcommand{\imageSchemaMerging}{}
%\newcommand{\imageSchemaSplitting}{}
\newcommand{\imageSchemaMomentum}{}
\newcommand{\imageSchemaSelfMotion}{}
%\newcommand{\imageSchemaBigSmall}{}
\newcommand{\imageSchemaFastSlow}{}
%\newcommand{\imageSchemaPartWhole}{}

\newcommand{\realworld}{}
\newcommand{\realworldNaivePhysic}{}
\newcommand{\realworldBodyAwareness}{}
\newcommand{\realworldEnvironmentAwareness}{}
\newcommand{\realworldSocialAwareness}{}

\newcommand{\metaphor}{}
\newcommand{\metaphordesc}{Jemandem ein Objekt zuwerfen (Frisbee, Ball).}

% -------------------------------
% TECHNISCHES
% -------------------------------

%\newcommand{\technologyObjectIntimate}{}
\newcommand{\technologyObjectPersonal}{}
\newcommand{\technologyObjectSocial}{}
\newcommand{\technologyObjectPublic}{}

\newcommand{\technologyObjectDesc}{...}

%\newcommand{\technologyCommunicationServer}{}
%\newcommand{\technologyCommunicationAdhoc}{}

\newcommand{\technologyCommunicationDesc}{...}

%\newcommand{\technologyOrientationAccelerometer}{}
%\newcommand{\technologyOrientationGPS}{}
%\newcommand{\technologyOrientationGyroskop}{}
%\newcommand{\technologyOrientationAnnaeherung}{}
%\newcommand{\technologyOrientationHoehe}{}
%\newcommand{\technologyOrientationBeacons}{}
%\newcommand{\technologyOrientationOther}{}

\newcommand{\technologyOrientationDesc}{...}

\newcommand{\prototype}{...}


% -------------------------------
% SONSTIGES
% -------------------------------

\newcommand{\authors}{Alexander Hahn, Hochschule Mannheim  \\
Valentina Burjan, Hochschule Mannheim \\
Dominick Madden, Hochschule Mannheim \\
Horst Schneider, Hochschule Mannheim}
\newcommand{\versionhistory}{06.03.2017}
\newcommand{\dateofcreation}{27.07.2016}
\newcommand{\comments}{---}
\newcommand{\questions}{---}


% template inkludieren --------------

\maketitle


\section*{Was}

\subsection*{Problem}
\desc

\subsection*{Lösung}
\solution

\subsection*{Grafische Darstellung}

\begin{figure}[H]
\includegraphics[scale=0.3]{mypicture.png}
\end{figure}


\subsection*{Kategorie}
\ifthenelse{\equal{\category}{give}}{$\boxtimes$}{$\Box$} Give \\
\ifthenelse{\equal{\category}{take}}{$\boxtimes$}{$\Box$} Take \\
\ifthenelse{\equal{\category}{exchange}}{$\boxtimes$}{$\Box$} Exchange \\
\ifthenelse{\equal{\category}{extend}}{$\boxtimes$}{$\Box$} Extend \\
\ifthenelse{\equal{\category}{connect}}{$\boxtimes$}{$\Box$} Connect



\section*{Wie}

\subsection*{Aktion des Benutzers}
\useraction

\subsection*{Reaktion des Sende-und Empfänger-Gerätes}
%\reaction

\subsection*{Hinweise zur Gestaltung der Interaktion}
%\designnotes



\section*{Wann}

\subsection*{Geeigneter Nutzungskontext}

\subsubsection*{Zeit}
\ifthenelse{\equal{\when}{gleichzeitig}}{$\boxtimes$}{$\Box$} gleichzeitige Nutzung von Geräten \\
\ifthenelse{\equal{\when}{aufeinanderfolgend}}{$\boxtimes$}{$\Box$} sequentielle Nutzung von Geräten 

\subsubsection*{Modus}
\ifthenelse{\equal{\mode}{online}}{$\boxtimes$}{$\Box$} online \\
\ifthenelse{\equal{\mode}{offline}}{$\boxtimes$}{$\Box$} offline \\

%\validcontext

\subsection*{Abzuratender Nutzungskontext}
%\notvalidcontext

\subsection*{Geräteklassen}
\begin{tabular}{|c|c|c|c|c|}
\hline 
• & • & Mittel & Riesig & Groß \\ 
\hline 
• & • & • & • & • \\ 
\hline 
• & • & • & • & • \\ 
\hline 
• & • & • & • & • \\ 
\hline 
• & • & • & • & • \\ 
\hline 
\end{tabular} 

\subsection*{Entfernung zwischen Sender- und Empfänger-Gerät}



\section*{Warum}


\subsection*{Displaygrößen}


\subsection*{Analoge Patterns}


\subsection*{State of the Art/Gebrauchshistorie}


\subsection*{Checkliste: Entspricht die Interaktion der Definiton eines "Blended Interaction"?}


\section*{Technisches}

\subsection*{Technologien zur Objekterkennung}


\subsection*{Technologien zur Kommunikation}


\subsection*{Technologien zur Bewegungs-/Orientierungsbestimmung}


\subsection*{Prototyp/Lösungsansatz/Code-Snippets/UML-Diagramm}



\section*{Sonstiges}

\subsection*{Autor/en}

\subsection*{Literaturreferenzen}

\subsection*{Abbildungsverzeichnis}

\subsection*{Versionshistorie}

\subsection*{Kommentare}

\subsection*{Offene Fragen}