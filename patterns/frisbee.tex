\documentclass[11pt,a4paper,notitlepage]{article}

\usepackage[utf8]{inputenc}
\usepackage[T1]{fontenc}
\usepackage[german]{babel}
\usepackage{float}
%\usepackage{amsmath}
%\usepackage{amsfonts}
\usepackage{amssymb}
\usepackage{graphicx}
\usepackage{ifthen}
\usepackage{diagbox}
\usepackage[hyphens]{url}
\usepackage{textcomp,gensymb}
\usepackage{makecell}
\usepackage{textpos}
\usepackage{tabularx}
\usepackage{csquotes}
%\usepackage{hyperref}
\usepackage[natbib=true,bibstyle=numeric,backend=bibtexu,citestyle=numeric]{biblatex}
\bibliography{lit.bib} 
\renewcommand\theadalign{cb}
\renewcommand\theadfont{\bfseries}
\renewcommand\theadgape{\Gape[4pt]}

\usepackage{nopageno}

\author{}
\date{}
\title{\name}

% Template für Checkboxen ----
\newcommand{\checkbox}[1]{
\ifx#1\undefined
  $\Box$
\else
  $\boxtimes$  
\fi}

\setlength{\parindent}{0pt}

%----------------------------

\newcommand{\grafischedarstellung}{\jobname_graphical_description.png}

%----------------------------

\newcommand{\umldiagram}{\jobname_uml.png}

%----------------------------

\newcommand{\sequencediagram}{\jobname_sequence.png}

%----------------------------

\newcommand{\solutionimg}{\jobname_solution.png}

%----------------------------

\newcommand{\prototypeimg}{\jobname_prototype.png}

% --------- Glossary
\newcommand{\sen}{Sender}
\newcommand{\rec}{Empfänger}
\newcommand{\recdev}{Empfangsgerät}
\newcommand{\sendev}{Sendegerät}
\newcommand{\data}{Datenobjekt}

\newcommand{\name}{Frisbee}

% -------------------------------
% WAS
% -------------------------------

\newcommand{\desc}{Ein Datenobjekt von einem Sender-Gerät soll auch auf einem Empfänger-Gerät verfügbar sein.}

\newcommand{\solution}{Der Benutzer hält Sender-Gerät in der Hand und schwenkt es waagerecht aus dem Handgelenk heraus in Richtung des Empfänger-Gerätes. Dabei wird ein Datenobjekt vom Quell-Gerät auf das Ziel-Gerät übertragen.}

\newcommand{\category}{give}
%\newcommand{\category}{take}
%\newcommand{\category}{exchange}
%\newcommand{\category}{extend}
%\newcommand{\category}{connect}

% -------------------------------
% WIE
% -------------------------------

\newcommand{\useraction}{Der Benutzer hält ein Sender-Gerät fest in der Hand. Aus dem Handgelenk heraus schwenkt der Benutzer das Gerät (vom Körper weg) wie eine Frisbee-Scheibe in Richtung des Empfänger-Gerätes.}

\newcommand{\reaction}{Das Sender-Gerät erkennt die Schwenkbewegung. Dadurch wird ein ausgewähltes Datenobjekt an das Empfänger-Gerät übertragen und durch eine akustische Rückmeldung oder z.B. Vibration bestätigt.\\
Das Ziel-Gerät zeigt nach dem Empfangen des Datenobjektes dieses auf dem Display an.
}

\newcommand{\reactionSuccessVisual}{}
\newcommand{\reactionSuccessAcustic}{}
\newcommand{\reactionSuccessSensitive}{}

\newcommand{\reactionFailureConnection}{}
\newcommand{\reactionFailureConnectionDesc}{muss dem Anwender durch ein Signal, z.B. einen Ton, mitgeteilt werden, dass die Interaktion und Datenübertragung nicht funktioniert.}

\newcommand{\reactionFailureNoDevice}{}
\newcommand{\reactionFailureNoDeviceDesc}{muss der Anwender ggf. die Interaktion erneut ausführen. Solange kein erfolgreicher Datentransfer stattgefunden hat, sollte dem Anwender signalisiert werden, dass das Zielgerät nicht erkannt wurde.}

\newcommand{\reactionFailureCompatibility}{}
\newcommand{\reactionFailureCompatibilityDesc}{sollte der Anwender vor erneuter Ausführung der Interaktion informiert werden, welche Geräte kompatibel sind und die Interaktion erneut ausführen.}

\newcommand{\designnotes}{
\begin{itemize}
\item Eine Rückmeldung nach dem Senden der Daten (nach der Schwenk-Bewegung) ist erforderlich, um dem Benutzer mitzuteilen, dass die Aktion erfolgreich war. Erfolgt keine Rückmeldung, besteht die Möglichkeit, dass der Benutzer häufig hintereinander die Interaktion durchführt, obwohl bereits die Daten erfolgreich das Ziel-Gerät versendet wurden.
\item Das Empfänger-Gerät sollte standfest sein (z.B. Screens an der Wand) und einen Abstand zum Sender-Gerät gewährleisten, um die Interaktion korrekt auszuführen.
\item Die Schwenk-Bewegung muss eine definierte Geschwindigkeit erreichen bis das System die Bewegung als Interaktion erkennt, damit der Anwender ein natürliches Empfinden hat bei der Ausführung der Interaktion.
\end{itemize}}

% -------------------------------
% WANN
% -------------------------------

\newcommand{\validcontext}{...}

%\newcommand{\simultaneously}{}
%\newcommand{\sequentially}{}

%\newcommand{\online}{}
%\newcommand{\offline}{}

%\newcommand{\private}{}
%\newcommand{\semipublic}{}
%\newcommand{\public}{}
%\newcommand{\stationary}{}
%\newcommand{\onthego}{}

%\newcommand{\leanback}{}
%\newcommand{\leanforward}{}

%\newcommand{\single}{}
%\newcommand{\collaboration}{}
%\newcommand{\facetoface}{}
%\newcommand{\sidetoside}{}
%\newcommand{\cornertocorner}{}

%\newcommand{\smalltask}{}
%\newcommand{\repeatedtask}{}
%\newcommand{\locationbased}{}
%\newcommand{\distraction}{}
%\newcommand{\urgent}{} 

\newcommand{\notvalidcontext}{...}


\newcommand{\devicetabular}{
\begin{tabular}[H]{|c|c|c|c|c|c|}
\hline 
\diagbox{von}{nach}   & Smartwatch & Smartphone & Tablet & Tabletop & Screens \\ 
\hline 
Smartwatch            &     •      &     •      &   •    &     •    &     •   \\ 
\hline 
Smartphone            &     •      &     •      &   •    &     •    &     •   \\ 
\hline 
Tablet                &     •      &     •      &   •    &     •    &     •   \\ 
\hline 
Tabletop              &     •      &     •      &   •    &     •    &     •   \\ 
\hline
Screens               &     •      &     •      &   •    &     •    &     •   \\ 
\hline 
\end{tabular} }

% -------------------------------
% WARUM
% -------------------------------

%\newcommand{\established}{}
%\newcommand{\candidate}{}
%\newcommand{\realizable}{}
%\newcommand{\futuristic}{}

\newcommand{\otherpatterns}{...}

\newcommand{\stateoftheart}{
%\begin{enumerate}
%\item
%\item
%\end{enumerate}
}


%\newcommand{\designprinciples}{}

%\newcommand{\imageschemata}{}
%\newcommand{\imageSchemaContainer}{}
%\newcommand{\imageSchemaInOut}{}
%\newcommand{\imageSchemaPath}{}
%\newcommand{\imageSchemaSourcePathGoal}{}
%\newcommand{\imageSchemaUpDown}{}
%\newcommand{\imageSchemaLeftRight}{}
%\newcommand{\imageSchemaNearFar}{}
%\newcommand{\imageSchemaPartWhole}{}

%\newcommand{\realworld}{}
%\newcommand{\realworldNaivePhysic}{}
%\newcommand{\realworldBodyAwareness}{}
%\newcommand{\realworldEnvironmentAwareness}{}
%\newcommand{\realworldSocialAwareness}{}

%\newcommand{\metaphor}{}
\newcommand{\metaphordesc}{...}

% -------------------------------
% TECHNISCHES
% -------------------------------

%\newcommand{\technologyObjectIntimate}{}
%\newcommand{\technologyObjectPersonal}{}
%\newcommand{\technologyObjectSocial}{}
%\newcommand{\technologyObjectPublic}{}

\newcommand{\technologyObjectDesc}{...}

%\newcommand{\technologyCommunicationServer}{}
%\newcommand{\technologyCommunicationAdhoc}{}

\newcommand{\technologyCommunicationDesc}{...}

%\newcommand{\technologyOrientationAccelerometer}{}
%\newcommand{\technologyOrientationGPS}{}
%\newcommand{\technologyOrientationGyroskop}{}
%\newcommand{\technologyOrientationAnnaeherung}{}
%\newcommand{\technologyOrientationHoehe}{}
%\newcommand{\technologyOrientationBeacons}{}
%\newcommand{\technologyOrientationOther}{}

\newcommand{\technologyOrientationDesc}{...}

\newcommand{\prototype}{...}


% -------------------------------
% SONSTIGES
% -------------------------------

\newcommand{\authors}{...}
\newcommand{\literature}{...}
\newcommand{\figures}{...}
\newcommand{\versionhistory}{...}
\newcommand{\dateofcreation}{...}
\newcommand{\comments}{...}
\newcommand{\questions}{...}


% template inkludieren --------------

\maketitle


\section*{Was}

\subsection*{Problem}
\desc

\subsection*{Lösung}
\solution

\subsection*{Grafische Darstellung}

\begin{figure}[H]
\includegraphics[scale=0.3]{mypicture.png}
\end{figure}


\subsection*{Kategorie}
\ifthenelse{\equal{\category}{give}}{$\boxtimes$}{$\Box$} Give \\
\ifthenelse{\equal{\category}{take}}{$\boxtimes$}{$\Box$} Take \\
\ifthenelse{\equal{\category}{exchange}}{$\boxtimes$}{$\Box$} Exchange \\
\ifthenelse{\equal{\category}{extend}}{$\boxtimes$}{$\Box$} Extend \\
\ifthenelse{\equal{\category}{connect}}{$\boxtimes$}{$\Box$} Connect



\section*{Wie}

\subsection*{Aktion des Benutzers}
\useraction

\subsection*{Reaktion des Sende-und Empfänger-Gerätes}
%\reaction

\subsection*{Hinweise zur Gestaltung der Interaktion}
%\designnotes



\section*{Wann}

\subsection*{Geeigneter Nutzungskontext}

\subsubsection*{Zeit}
\ifthenelse{\equal{\when}{gleichzeitig}}{$\boxtimes$}{$\Box$} gleichzeitige Nutzung von Geräten \\
\ifthenelse{\equal{\when}{aufeinanderfolgend}}{$\boxtimes$}{$\Box$} sequentielle Nutzung von Geräten 

\subsubsection*{Modus}
\ifthenelse{\equal{\mode}{online}}{$\boxtimes$}{$\Box$} online \\
\ifthenelse{\equal{\mode}{offline}}{$\boxtimes$}{$\Box$} offline \\

%\validcontext

\subsection*{Abzuratender Nutzungskontext}
%\notvalidcontext

\subsection*{Geräteklassen}
\begin{tabular}{|c|c|c|c|c|}
\hline 
• & • & Mittel & Riesig & Groß \\ 
\hline 
• & • & • & • & • \\ 
\hline 
• & • & • & • & • \\ 
\hline 
• & • & • & • & • \\ 
\hline 
• & • & • & • & • \\ 
\hline 
\end{tabular} 

\subsection*{Entfernung zwischen Sender- und Empfänger-Gerät}



\section*{Warum}


\subsection*{Displaygrößen}


\subsection*{Analoge Patterns}


\subsection*{State of the Art/Gebrauchshistorie}


\subsection*{Checkliste: Entspricht die Interaktion der Definiton eines "Blended Interaction"?}


\section*{Technisches}

\subsection*{Technologien zur Objekterkennung}


\subsection*{Technologien zur Kommunikation}


\subsection*{Technologien zur Bewegungs-/Orientierungsbestimmung}


\subsection*{Prototyp/Lösungsansatz/Code-Snippets/UML-Diagramm}



\section*{Sonstiges}

\subsection*{Autor/en}

\subsection*{Literaturreferenzen}

\subsection*{Abbildungsverzeichnis}

\subsection*{Versionshistorie}

\subsection*{Kommentare}

\subsection*{Offene Fragen}