\documentclass[11pt,a4paper,notitlepage]{article}

\usepackage[utf8]{inputenc}
\usepackage[T1]{fontenc}
\usepackage[german]{babel}
\usepackage{float}
%\usepackage{amsmath}
%\usepackage{amsfonts}
\usepackage{amssymb}
\usepackage{graphicx}
\usepackage{ifthen}
\usepackage{diagbox}
\usepackage[hyphens]{url}
\usepackage{textcomp,gensymb}
\usepackage{makecell}
\usepackage{textpos}
\usepackage{tabularx}
\usepackage{csquotes}
%\usepackage{hyperref}
\usepackage[natbib=true,bibstyle=numeric,backend=bibtexu,citestyle=numeric]{biblatex}
\bibliography{lit.bib} 
\renewcommand\theadalign{cb}
\renewcommand\theadfont{\bfseries}
\renewcommand\theadgape{\Gape[4pt]}

\usepackage{nopageno}

\author{}
\date{}
\title{\name}

% Template für Checkboxen ----
\newcommand{\checkbox}[1]{
\ifx#1\undefined
  $\Box$
\else
  $\boxtimes$  
\fi}

\setlength{\parindent}{0pt}

%----------------------------

\newcommand{\grafischedarstellung}{\jobname_graphical_description.png}

%----------------------------

\newcommand{\umldiagram}{\jobname_uml.png}

%----------------------------

\newcommand{\sequencediagram}{\jobname_sequence.png}

%----------------------------

\newcommand{\solutionimg}{\jobname_solution.png}

%----------------------------

\newcommand{\prototypeimg}{\jobname_prototype.png}

% --------- Glossary
\newcommand{\sen}{Sender}
\newcommand{\rec}{Empfänger}
\newcommand{\recdev}{Empfangsgerät}
\newcommand{\sendev}{Sendegerät}
\newcommand{\data}{Datenobjekt}

\newcommand{\name}{Pinch To Connect}

% -------------------------------
% WAS
% -------------------------------

\newcommand{\desc}{Ein Benutzer (der \sen{}) möchte sein Gerät (das \sendev{}) mit einem anderen, in unmittelbarer Nähe befindlichen Gerät (dem \recdev{}) verbinden um eine Datenübertragung zu ermöglichen.}

\newcommand{\solution}{Der \sen\ hat das \sendev\ Kante an Kante neben dem \recdev\ und führt eine \glslink{pinch}{Pinch-Geste} aus, wodurch der Verbindungsvorgang angestoßen wird.}

%\newcommand{\category}{give}
%\newcommand{\category}{take}
%\newcommand{\category}{exchange}
%\newcommand{\category}{extend}
\newcommand{\category}{connect}

% -------------------------------
% WIE
% -------------------------------

\newcommand{\useraction}{Der \sen\ hat das \sendev\ entweder auf einer Oberfläche - Kante an Kante - neben dem \recdev\ liegen, oder hat es in der Hand und hält es neben das \recdev{}. Er führt eine Pinch-Geste aus, wobei sich der Daumen auf dem Display des einen Geräts befindet und der Zeigefinger auf dem des anderen.}

\newcommand{\reactionSen}{Auf beiden Geräten sollte zu verschiedenen Phasen des Pinchs Feedback gegeben werden. Ein Pinch besteht aus zwei simultanen Swipes, jeweils von der Bildschirmmitte der Geräte zum Rand. Jeder der beiden Swipes besteht aus den drei \glslink{atomareinteraktion}{atomaren Interaktionen} \textit{Touch}, \textit{Move} und \textit{Release}.\\
Da es sich bei \textbf{Pinch to Connect} um eine \gls{synchronegeste} handelt, werden nach dem Pinch einige relevanten Daten (Richtung, Timestamp, Orientierung) der beiden Swipes verglichen. Entsprechen die Daten einem Pinch, so wird der Verbindungsvorgang angestoßen. \\
Vor dem Pinch besteht keine direkte Verbindung zwischen den Geräten, daher muss es eine \gls{vermittlungskomponente} geben (z.B. ein externer Server), an die die Daten zur Überprüfung gesendet werden. Sind die Daten ähnlich genug, gilt der Pinch als \textit{erfolgreich}. Bei einem \textit{erfolgreich} ausgeführten Pinch finden alle atomaren Interaktionen statt, nach Beendigung der letzten werden die Geräte drahtlos verbunden (Atomare Interaktion \textit{Connect}), worüber der Nutzer ebenfalls Feedback erhalten sollte.}

\newcommand{\reactionRec}{Handelt es sich bei dem \recdev\ um ein mobiles Gerät, das von einem Nutzer gehalten wird, werden die selben atomaren Interaktionen ausgeführt wie auf dem \sendev .\
}

\newcommand{\microinteractionstabular}{
\begin{figure}[H]
\begin{table}[H]
\renewcommand{\arraystretch}{2}\addtolength{\tabcolsep}{-2pt}
\centering
\newcolumntype{b}{X}
\newcolumntype{t}{>{\hsize=.3\hsize}X}
\newcolumntype{s}{>{\hsize=.2\hsize}c}
\newcolumntype{m}{>{\hsize=.6\hsize}X}
\begin{tabularx}{\textwidth}{tsbbm}
\thead[X]{Name} & \thead[c]{Typ*} & \thead[X]{Trigger} & \thead[X]{Regeln} & \thead[X]{Feedback} \\
\hline
Touch & M & Touch Down Event auf dem Screen & Touch auf dem \data\ &  Animation 1 \\ 
\hline
Move & M & Touch Move Event auf dem Screen & 
Touch ausgeführt, \newline Release nicht ausgeführt & Animation 2 \\ 
\hline
Release & M & Touch Up Event auf dem Screen & Swipelänge OK, \newline Swipedauer OK, \newline Swipeorientierung OK & Animation 3 \\ 
\hline
Connect & S & Verbindung zwischen den Geräten wurde hergestellt & Datentransfer \newline ist möglich & Animation 4 \\
\hline
\end{tabularx}
\end{table}
\caption{Atomare Interaktionen für das Pinch to Connect Pattern}
\end{figure}
*Typ: (M)anuell, (S)ystem
}

\newcommand{\animations}{
\begin{enumerate}
\item Touch-Animation: visualisiert dem Benutzer, dass ein Pinch die Verbindungsfunktionalität auslöst (z.B. Ein Gerät zeigt eine Steckdose an, das andere einen Stecker. Führt man sie zusammen, so verbindet man die Geräte miteinander)
\item Move-Animation: visualisiert dem Benutzer, dass die Objekte beweglich sind (z.B. Drag-and-Drop von Stecker und Steckdose)
\item Release-Animation: visualisiert dem Benutzer, dass Geste erfolgreich ausgeführt wurde und ein Verbindungsaufbau initiiert wird (z.B. Stecker und Steckdose verbinden sich)
\item Connect-Animation: visualisiert dem Benutzer, dass die Verbindung hergestellt wurde (z.B. Es fließen Daten durch die Kabel des Steckers und der Steckdose)
\end{enumerate}
}

\newcommand{\designnotes}{
\begin{itemize}
\item[-] Beide involvierten Geräte benötigen einen Touchscreen.
\item[-] Bei jeder Synchronen Connect-Geste muss es eine Vermittlungskomponente geben, an die die relevanten Daten der Geste gesendet werden. Diese vergleicht die empfangenen Daten der beiden involvierten Geräte und gibt positive oder negative Rückmeldung.
\end{itemize}}

% -------------------------------
% WANN
% -------------------------------

\newcommand{\validcontext}{Verbinden von mobilen Geräten zwecks Datenübertragung, Verbinden von mobilen Privatgeräten mit stationären Geräten (z.B. Verbindung zu einem Netzwerk am Arbeitsplatz)}

\newcommand{\simultaneously}{}
%\newcommand{\sequentially}{}

\newcommand{\online}{}
\newcommand{\offline}{}

\newcommand{\private}{}
\newcommand{\semipublic}{}
\newcommand{\public}{}
\newcommand{\stationary}{}
\newcommand{\onthego}{}

%\newcommand{\leanback}{}
\newcommand{\leanforward}{}

\newcommand{\single}{}
%\newcommand{\collaboration}{}
%\newcommand{\facetoface}{}
\newcommand{\sidetoside}{}
%\newcommand{\cornertocorner}{}

\newcommand{\notvalidcontext}{Sichtbarmachen vertraulicher Informationen (z.B. Name, Alter etc.) auf öffentlichen Displays.}


\newcommand{\devicetabular}{
\begin{tabular}[H]{|c|c|c|c|c|c|}
\hline 
\diagbox{von}{nach}   & Smartwatch & Smartphone & Tablet & Tabletop & Screens \\ 
\hline 
Smartwatch            &            &            &        &          &         \\ 
\hline 
Smartphone            &            &     x      &   x    &     x    &        \\ 
\hline 
Tablet                &            &     x      &   x    &     x    &    
\\ 
\hline 
Tabletop              &            &            &        &          &         \\ 
\hline
Screens               &            &            &        &          &         \\ 
\hline 
\end{tabular}}

% -------------------------------
% WARUM
% -------------------------------

%\newcommand{\established}{}
\newcommand{\candidate}{}
\newcommand{\realizable}{}
%\newcommand{\futuristic}{}

\newcommand{\otherpatterns}{
\begin{itemize}
\item Spread To Disconnect
\item Stitch To Give
\item Stitch To Take
\item Stitch To Extend
\end{itemize}
}

\newcommand{\stateoftheart}{
\begin{enumerate}
%\item Bump App: Bis 2014 in den App/Play Stores erhältlich gewesen [\url{http://bu.mp/}]
%\item Beispiel-Implementierung eines Bump Patterns \citep{Grab2015}.
%\item Microinteractions im Multi-Screen Kontext \citep{Madden2016}.
\item Stitch-to-Tile: Eine Gestensteuerung zur Auslösung von Bildschirmerweiterung \citep{Madden2015}
\end{enumerate}
}

\newcommand{\designprinciples}{}

\newcommand{\imageschemata}{}
%\newcommand{\imageSchemaVoid}{}
%\newcommand{\imageSchemaObject}{}
%\newcommand{\imageSchemaSubstance}{}
%\newcommand{\imageSchemaCenterPeriphery}{}
%\newcommand{\imageSchemaContact}{}
%\newcommand{\imageSchemaFrontBack}{}
%\newcommand{\imageSchemaLocation}{}
%\newcommand{\imageSchemaNearFar}{}
\newcommand{\imageSchemaPath}{}
%\newcommand{\imageSchemaSourcePathGoal}{}
%\newcommand{\imageSchemaScale}{}
\newcommand{\imageSchemaLeftRight}{}
%\newcommand{\imageSchemaContainer}{}
%\newcommand{\imageSchemaContent}{}
%\newcommand{\imageSchemaFullEmpty}{}
%\newcommand{\imageSchemaInOut}{}
%\newcommand{\imageSchemaSurface}{}
%\newcommand{\imageSchemaMerging}{}
%\newcommand{\imageSchemaSplitting}{}
%\newcommand{\imageSchemaMomentum}{}
%\newcommand{\imageSchemaSelfMotion}{}
%\newcommand{\imageSchemaBigSmall}{}
%\newcommand{\imageSchemaFastSlow}{}
%\newcommand{\imageSchemaPartWhole}{}

\newcommand{\realworld}{}
\newcommand{\realworldNaivePhysic}{}
\newcommand{\realworldBodyAwareness}{}
\newcommand{\realworldEnvironmentAwareness}{}
\newcommand{\realworldSocialAwareness}{}

\newcommand{\metaphor}{}
\newcommand{\metaphordesc}{Zusammenführen von Stecker und Steckdose}

% -------------------------------
% TECHNISCHES
% -------------------------------

\newcommand{\technologyObjectIntimate}{}
\newcommand{\technologyObjectPersonal}{}
%\newcommand{\technologyObjectSocial}{}
%\newcommand{\technologyObjectPublic}{}

\newcommand{\technologyObjectDesc}{}

%\newcommand{\technologyCommunicationServer}{}
\newcommand{\technologyCommunicationAdhoc}{}

\newcommand{\technologyCommunicationDesc}{}

%\newcommand{\technologyOrientationAccelerometer}{}
%\newcommand{\technologyOrientationGPS}{}
%\newcommand{\technologyOrientationGyroskop}{}
%\newcommand{\technologyOrientationAnnaeherung}{}
%\newcommand{\technologyOrientationHoehe}{}
%\newcommand{\technologyOrientationBeacons}{}
%\newcommand{\technologyOrientationOther}{}

\newcommand{\technologyOrientationDesc}{-}

\newcommand{\prototype}{...}


% -------------------------------
% SONSTIGES
% -------------------------------

\newcommand{\authors}{
Benjamin Grab, Hochschule Mannheim\\
Valentina Burjan, Hochschule Mannheim\\
Horst Schneider, Hochschule Mannheim\\
Dominick Madden, Hochschule Mannheim}

\newcommand{\literature}{
\begin{enumerate}
\item Ken Hinckley. 2003. Synchronous gestures for multiple persons and computers. In Proceedings of the 16th annual ACM symposium on User interface software and technology (UIST '03). ACM, New York, NY, USA, 149-158. DOI=10.1145/964696.964713
\end{enumerate}
}
\newcommand{\figures}{...}
\newcommand{\versionhistory}{...}
\newcommand{\dateofcreation}{...}
\newcommand{\comments}{...}
\newcommand{\questions}{...}


% template inkludieren --------------

\maketitle


\section*{Was}

\subsection*{Problem}
\desc

\subsection*{Lösung}
\solution

\subsection*{Grafische Darstellung}

\begin{figure}[H]
\includegraphics[scale=0.3]{mypicture.png}
\end{figure}


\subsection*{Kategorie}
\ifthenelse{\equal{\category}{give}}{$\boxtimes$}{$\Box$} Give \\
\ifthenelse{\equal{\category}{take}}{$\boxtimes$}{$\Box$} Take \\
\ifthenelse{\equal{\category}{exchange}}{$\boxtimes$}{$\Box$} Exchange \\
\ifthenelse{\equal{\category}{extend}}{$\boxtimes$}{$\Box$} Extend \\
\ifthenelse{\equal{\category}{connect}}{$\boxtimes$}{$\Box$} Connect



\section*{Wie}

\subsection*{Aktion des Benutzers}
\useraction

\subsection*{Reaktion des Sende-und Empfänger-Gerätes}
%\reaction

\subsection*{Hinweise zur Gestaltung der Interaktion}
%\designnotes



\section*{Wann}

\subsection*{Geeigneter Nutzungskontext}

\subsubsection*{Zeit}
\ifthenelse{\equal{\when}{gleichzeitig}}{$\boxtimes$}{$\Box$} gleichzeitige Nutzung von Geräten \\
\ifthenelse{\equal{\when}{aufeinanderfolgend}}{$\boxtimes$}{$\Box$} sequentielle Nutzung von Geräten 

\subsubsection*{Modus}
\ifthenelse{\equal{\mode}{online}}{$\boxtimes$}{$\Box$} online \\
\ifthenelse{\equal{\mode}{offline}}{$\boxtimes$}{$\Box$} offline \\

%\validcontext

\subsection*{Abzuratender Nutzungskontext}
%\notvalidcontext

\subsection*{Geräteklassen}
\begin{tabular}{|c|c|c|c|c|}
\hline 
• & • & Mittel & Riesig & Groß \\ 
\hline 
• & • & • & • & • \\ 
\hline 
• & • & • & • & • \\ 
\hline 
• & • & • & • & • \\ 
\hline 
• & • & • & • & • \\ 
\hline 
\end{tabular} 

\subsection*{Entfernung zwischen Sender- und Empfänger-Gerät}



\section*{Warum}


\subsection*{Displaygrößen}


\subsection*{Analoge Patterns}


\subsection*{State of the Art/Gebrauchshistorie}


\subsection*{Checkliste: Entspricht die Interaktion der Definiton eines "Blended Interaction"?}


\section*{Technisches}

\subsection*{Technologien zur Objekterkennung}


\subsection*{Technologien zur Kommunikation}


\subsection*{Technologien zur Bewegungs-/Orientierungsbestimmung}


\subsection*{Prototyp/Lösungsansatz/Code-Snippets/UML-Diagramm}



\section*{Sonstiges}

\subsection*{Autor/en}

\subsection*{Literaturreferenzen}

\subsection*{Abbildungsverzeichnis}

\subsection*{Versionshistorie}

\subsection*{Kommentare}

\subsection*{Offene Fragen}
