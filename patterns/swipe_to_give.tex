\documentclass[11pt,a4paper,notitlepage]{article}

\usepackage[utf8]{inputenc}
\usepackage[T1]{fontenc}
\usepackage[german]{babel}
\usepackage{float}
%\usepackage{amsmath}
%\usepackage{amsfonts}
\usepackage{amssymb}
\usepackage{graphicx}
\usepackage{ifthen}
\usepackage{diagbox}
\usepackage[hyphens]{url}
\usepackage{textcomp,gensymb}
\usepackage{makecell}
\usepackage{textpos}
\usepackage{tabularx}
\usepackage{csquotes}
%\usepackage{hyperref}
\usepackage[natbib=true,bibstyle=numeric,backend=bibtexu,citestyle=numeric]{biblatex}
\bibliography{lit.bib} 
\renewcommand\theadalign{cb}
\renewcommand\theadfont{\bfseries}
\renewcommand\theadgape{\Gape[4pt]}

\usepackage{nopageno}

\author{}
\date{}
\title{\name}

% Template für Checkboxen ----
\newcommand{\checkbox}[1]{
\ifx#1\undefined
  $\Box$
\else
  $\boxtimes$  
\fi}

\setlength{\parindent}{0pt}

%----------------------------

\newcommand{\grafischedarstellung}{\jobname_graphical_description.png}

%----------------------------

\newcommand{\umldiagram}{\jobname_uml.png}

%----------------------------

\newcommand{\sequencediagram}{\jobname_sequence.png}

%----------------------------

\newcommand{\solutionimg}{\jobname_solution.png}

%----------------------------

\newcommand{\prototypeimg}{\jobname_prototype.png}

% --------- Glossary
\newcommand{\sen}{Sender}
\newcommand{\rec}{Empfänger}
\newcommand{\recdev}{Empfangsgerät}
\newcommand{\sendev}{Sendegerät}
\newcommand{\data}{Datenobjekt}

\newcommand{\name}{Swipe To Give}

% -------------------------------
% WAS
% -------------------------------

\newcommand{\desc}{Ein Benutzer (der \sen) hat ein \data\ und will es mit einem weiteren Benutzer (dem \rec) bzw. einem \recdev\ teilen. Es soll über Distanz vom \sendev\ auf das \recdev\ übertragen werden, sodass es anschließend auf beiden Geräten verfügbar ist.}

\newcommand{\solution}{Der \sen\ hat das \sendev\ in der Hand oder vor sich liegen. Durch eine Wischbewegung (Swipe) in Richtung des \recdev s\ wird das \data\ auf das \recdev\ übertragen.}

\newcommand{\category}{give}
%\newcommand{\category}{take}
%\newcommand{\category}{exchange}
%\newcommand{\category}{extend}
%\newcommand{\category}{connect}

% -------------------------------
% WIE
% -------------------------------

\newcommand{\useraction}{Der \sen\ hält das \sendev\ in der Hand oder hat es vor sich liegen bzw. stehen (z.B. ein Tablet oder Tabletop). Ein \data\ ist entweder explizit vom \sen\ (z.B. ein Foto) oder implizit durch die Applikation (z.B. der aktuelle Bildschirm) zum Versenden ausgewählt worden.\\
Der \sen\ führt einen Swipe mit einem Finger in Richtung des \recdev s\ aus. Der Swipe muss mit einer Dynamik versehen werden, die dem Sender das Gefühl gibt, die physikalische Distanz zum \recdev\ durch Bewegung des \data s\ zu überbrücken. Diese Dynamik kann über Anpassung der Parameter (Länge, Richtung, Geschwindigkeit) des Swipes beeinflusst werden.}

\newcommand{\reaction}{Ist der Datensatz erfolgreich versendet worden, so reagiert das Sendergerät mit einer visuellen Nachricht an den Benutzer. Er bekommt eine kleine Nachricht angezeigt, dass die Übertragung erfolgreich verlief (z.B. durch ein Ausfaden/Verblassen des Bildes).\\
Das Empfänger-Gerät empfängt den Datensatz und zeigt diesen auf dem Bildschirm an.\\
Die Reaktion beim Sender ist eine Popup-Nachricht, welche signalisiert, dass die Dateien übertragen wurden. Des Weiteren wird die Liste der ausgewählten Bilder geleert.\\
Die eigentliche Reaktion ist es die übertragenen Dateien anzuzeigen (visuelle Rückmeldung). Dies könnte bspw. durch Hineingleiten vom oberen Rand des Empfänger-Geräts sein, wie es auch beim Senden beschrieben wurde. Der Benutzer erhält durch diese gestalterische Lösung ein besseres Verständnis für den Vorgang. Das Gerät soll jedoch zusätzlich eine akustische Rückmeldung an den Benutzer geben, dass Daten übertragen wurden. In diesem Fall ist die akustische Rückmeldung ein Vibrationsalarm.}

\newcommand{\reactionSuccessVisual}{}
%\newcommand{\reactionSuccessAcustic}{}
\newcommand{\reactionSuccessSensitive}{}

\newcommand{\reactionFailureConnection}{}
\newcommand{\reactionFailureConnectionDesc}{wird ein Fehlertext auf dem Sender-Gerät angezeigt.}

%\newcommand{\reactionFailureNoDevice}{}
\newcommand{\reactionFailureNoDeviceDesc}{...}

%\newcommand{\reactionFailureCompatibility}{}
\newcommand{\reactionFailureCompatibilityDesc}{...}

\newcommand{\designnotes}{Entsprechend der Geräte ist es notwendig zu berücksichtigen, mit wie viel Fingern die Swipe-Geste ausgeführt werden soll. In diesem Fall wird der Swipe mit einem Finger ausgeführt und findet in einer eigens dafür gestalteten App statt. Um die Interaktion, den Swipe, durchführen zu können, wird eine Verbindung zu einem Gerät benötigt. }

\newcommand{\microinteractions}{
\begin{enumerate}
\item Touch
\item Move
\item Release
\item Receive
\end{enumerate}
}

\newcommand{\microinteractionstabular}{
\begin{tabular}[H]{|c|c|c|c|c|c|}
\hline 
\thead{N\degree} & \thead{Name} & \thead{Trigger Type} & \thead{Trigger} & \thead{Rules} & \thead{Feedback} \\ 
\hline 
1 &  Touch & Manual & Touch Event on Screen (Down) & -Touch must be on image &  Animation 1 \\ 
\hline
2 & Move & Manual & Touch Event on Screen (Move) & \makecell{-N\degree 1 has occured \\-N\degree 2 has not occured} & Animation 2 \\ 
\hline 
3 & Release & Manual & Touch Event on Screen (Up) &  \makecell{-Swipelength OK \\ -Swipeduration OK \\ -Swipeorientation OK} & Animation 3 \\ 
\hline 
4 & Receive & System & Data Received & -Data is an image file & \makecell{-Animation 4 \\ -Vibrate device} \\
\hline 
\end{tabular}}

\newcommand{\animations}{
\begin{enumerate}
\item Touch-Animation
\item Move-Animation
\item Release-Animation
\item Receive-Animation
\end{enumerate}
}

% -------------------------------
% WANN
% -------------------------------

\newcommand{\validcontext}{Datenaustausch von Bildern, Videos, Dateien, Social Network IDs}

%\newcommand{\simultaneously}{}
\newcommand{\sequentially}{}

\newcommand{\online}{}
\newcommand{\offline}{}

\newcommand{\private}{}
\newcommand{\semipublic}{}
\newcommand{\public}{}
\newcommand{\stationary}{}
\newcommand{\onthego}{}

\newcommand{\leanback}{}
\newcommand{\leanforward}{}

\newcommand{\single}{}
\newcommand{\collaboration}{}
\newcommand{\facetoface}{}
%\newcommand{\sidetoside}{}
%\newcommand{\cornertocorner}{}

\newcommand{\smalltask}{}
\newcommand{\repeatedtask}{}
%\newcommand{\locationbased}{}
%\newcommand{\distraction}{}
%\newcommand{\urgent}{} 

\newcommand{\notvalidcontext}{--- keine Information ---}


\newcommand{\devicetabular}{
\begin{tabular}[H]{|c|c|c|c|c|c|}
\hline 
\diagbox{von}{nach} & Smartwatch & Smartphone & Tablet & Tabletop & Screens \\ 
\hline 
Smartwatch          &            &     x      &   x    &     x    &     x   \\ 
\hline 
Smartphone          &            &     x      &   x    &     x    &     x   \\ 
\hline 
Tablet              &            &            &   x    &     x    &     x   \\ 
\hline 
Tabletop            &            &            &        &          &     x   \\ 
\hline
Screens             &            &            &        &          &         \\ 
\hline 
\end{tabular} }

% -------------------------------
% WARUM
% -------------------------------

%\newcommand{\established}{}
\newcommand{\candidate}{}
\newcommand{\realizable}{}
%\newcommand{\futuristic}{}

\newcommand{\otherpatterns}{--- keine Information ---}

\newcommand{\stateoftheart}{
\begin{enumerate}

\item \url{https://www.youtube.com/watch?v=GDdPN6mVLPM} \\
von Minute 0:56 bis 0:59\\
Mit der Swipe-Geste auf vom Gerät, wird das konfigurierte Auto auf die
Leinwand gegenüber des Kunden projiziert.

\item \url{http://vimeo.com/53606494} \\
von Minute 2:14 bis 2:25 \\
Mit der Swipe-Geste auf dem Tablet simuliert der Verkäufer (rotes
Hemd) dem Kunden (blaues Hemd) wie diverse Krawatten an ihm aussehen.

\item \url{http://www.microsoft.com/office/vision/} \\
von Minute 3:41 bis 3:45

\item \url{https://www.youtube.com/watch?v=ho00x7ZvDLw} \\
von Minute 0:35 bis 0:37

\item \url{https://www.youtube.com/watch?v=o_hKFOQolIg} \\
von Minute 1:04 bis 1:09 und 1:30 bis 1:45 in kollaborativer Nutzung

\item \url{http://vimeo.com/110928116} \\
von Minute 1:09 bis 1:15

\item \url{https://www.youtube.com/watch?v=yw564p8oF1M} \\
von Minute 0:14 bis 0:20 \\
Chromecast: Technologie mit der Daten z.B. vom Smartphone auf den
Fernseher übertragen werden.

\item \cite{Marquardt:2012:GEF:2396636.2396642}
\item \cite{Hoccer}
\item \cite{Sokolov:2012}
\item \cite{Fotoswipe}

\end{enumerate}
}

\newcommand{\designprinciples}{}

\newcommand{\imageschemata}{}
\newcommand{\imageSchemaContainer}{}
\newcommand{\imageSchemaInOut}{}
%\newcommand{\imageSchemaPath}{}
\newcommand{\imageSchemaSourcePathGoal}{}
%\newcommand{\imageSchemaUpDown}{}
%\newcommand{\imageSchemaLeftRight}{}
\newcommand{\imageSchemaNearFar}{}
%\newcommand{\imageSchemaPartWhole}{}

\newcommand{\realworld}{}
\newcommand{\realworldNaivePhysic}{}
\newcommand{\realworldBodyAwareness}{}
\newcommand{\realworldEnvironmentAwareness}{}
\newcommand{\realworldSocialAwareness}{}

\newcommand{\metaphor}{}
\newcommand{\metaphordesc}{Jemandem etwas aushändigen / übergeben}

% -------------------------------
% TECHNISCHES
% -------------------------------

%\newcommand{\technologyObjectIntimate}{}
%\newcommand{\technologyObjectPersonal}{}
%\newcommand{\technologyObjectSocial}{}
\newcommand{\technologyObjectPublic}{}

\newcommand{\technologyObjectDesc}{Aufgrund der Erhebung der technischen Vor- und Nachteile (TODO QUELLE HAHN BA), fiel die Auswahl der Übertragungstechnologie für den Prototyp auf WiFi Direct. Die hohe Reichweite und Übertragungsgeschwindigkeit ist für diese Entscheidung ausschlaggebend. Eine manuelle Auswahl des Zielgeräts ist für den Prototyp von Nachteil.\\

Durch die Methode discoverPeers() wird der Applikation mitgeteilt, dass nach
verfügbaren Geräten gesucht werden soll. Sobald Geräte gefunden werden,
wird der peerListListener aufgerufen. Dieser muss zuvor überschrieben werden,
damit auf die Liste zugegriffen werden kann. Sobald
neue Geräte gefunden werden, wird die alte Liste geleert und mit den Neuen
gefüllt. Daraufhin wird die GUI geupdated.}

%\newcommand{\technologyCommunicationServer}{}
\newcommand{\technologyCommunicationAdhoc}{}

\newcommand{\technologyCommunicationDesc}{
Um Dateien in Android auswählen zu können, wird das Storage Access Framework
(SAF) benötigt. Hierdurch können Nutzer ihre Bibliothek durchsuchen und
Dokumente, Bilder, sowie andere Dateien öffnen.}

%\newcommand{\technologyOrientationAccelerometer}{}
%\newcommand{\technologyOrientationGPS}{}
%\newcommand{\technologyOrientationGyroskop}{}
%\newcommand{\technologyOrientationAnnaeherung}{}
%\newcommand{\technologyOrientationHoehe}{}
%\newcommand{\technologyOrientationBeacons}{}
%\newcommand{\technologyOrientationOther}{}

\newcommand{\technologyOrientationDesc}{In Android werden Gesten vom Betriebssystem erkannt und können weiter verarbeitet
werden. Beispielsweise wird erkannt, wann und wo ein Finger das Display
berührt (Action Down). Das Gleiche gilt auch für das wieder Anheben des
Fingers (Action Up). Durch diese beiden Gesten kann ein Swipe erkannt werden.
Für eine Implementierung werden zwei Variablen benötigt.
Variable y1 steht für das Berühren des Displays, Variable y2 für das
wieder loslassen. Nun können die beiden Punkte und deren Entfernung gemessen
werden.\\
Ein Swipe hat eine Richtung, sowie eine Länge. In diesem Fall muss der Swipe
von unten nach oben erfolgen und die Länge muss mindestens 450 Pixel lang
sein.}

\newcommand{\prototype}{Um die Interaktion umzusetzen, müssen zwei Apps entwickelt werden, eine für das Start- und eine für das Zielgerät. Die App ist notwendig, damit das Swipen zur Datenübertragung genutzt werden kann. Außerdem müssen die Auswahl der Dateien und das Herstellen der Verbindung von Start- zu Zielgerät speziell für diese Interaktion angepasst werden.\\
Bei der Verbindung der beiden Endgeräte muss ein Informationskanal geöffnet werden, damit Dateien übertragen werden können. Des Weiteren muss mindestens ein Gerät erkennen, welche Geräte zur Interaktion zu Verfügung stehen.\\
Eine Applikation wurde auf der Android-Plattform implementiert mit der sich Endgeräte über WiFi Direct identifizieren und über lokale WLAN Netzwerke oder AD-Hoc Netzwerke Bilddateien austauschen können.}


% -------------------------------
% SONSTIGES
% -------------------------------

\newcommand{\authors}{
Alexander Hahn, Hochschule Mannheim  \\
Valentina Burjan // Hochschule Mannheim}
\newcommand{\literature}{
\begin{enumerate}
\item \url{https://github.com/informatik-mannheim/thesis-swipe-to-give/tree/master/sources/Code}\\
Java-Code zu den Demonstrator Applikationen von swipe-to-give

\item \url{https://github.com/informatik-mannheim/thesis-swipe-to-give/blob/master/sources/Video/SwipeToGive_Demo.mp4}\\
Demo-Video zu swipe-to-give
\end{enumerate}}
\newcommand{\figures}{---}
\newcommand{\versionhistory}{---}
\newcommand{\dateofcreation}{17.09.2015}
\newcommand{\comments}{---}
\newcommand{\questions}{---}


% template inkludieren --------------

\maketitle


\section*{Was}

\subsection*{Problem}
\desc

\subsection*{Lösung}
\solution

\subsection*{Grafische Darstellung}

\begin{figure}[H]
\includegraphics[scale=0.3]{mypicture.png}
\end{figure}


\subsection*{Kategorie}
\ifthenelse{\equal{\category}{give}}{$\boxtimes$}{$\Box$} Give \\
\ifthenelse{\equal{\category}{take}}{$\boxtimes$}{$\Box$} Take \\
\ifthenelse{\equal{\category}{exchange}}{$\boxtimes$}{$\Box$} Exchange \\
\ifthenelse{\equal{\category}{extend}}{$\boxtimes$}{$\Box$} Extend \\
\ifthenelse{\equal{\category}{connect}}{$\boxtimes$}{$\Box$} Connect



\section*{Wie}

\subsection*{Aktion des Benutzers}
\useraction

\subsection*{Reaktion des Sende-und Empfänger-Gerätes}
%\reaction

\subsection*{Hinweise zur Gestaltung der Interaktion}
%\designnotes



\section*{Wann}

\subsection*{Geeigneter Nutzungskontext}

\subsubsection*{Zeit}
\ifthenelse{\equal{\when}{gleichzeitig}}{$\boxtimes$}{$\Box$} gleichzeitige Nutzung von Geräten \\
\ifthenelse{\equal{\when}{aufeinanderfolgend}}{$\boxtimes$}{$\Box$} sequentielle Nutzung von Geräten 

\subsubsection*{Modus}
\ifthenelse{\equal{\mode}{online}}{$\boxtimes$}{$\Box$} online \\
\ifthenelse{\equal{\mode}{offline}}{$\boxtimes$}{$\Box$} offline \\

%\validcontext

\subsection*{Abzuratender Nutzungskontext}
%\notvalidcontext

\subsection*{Geräteklassen}
\begin{tabular}{|c|c|c|c|c|}
\hline 
• & • & Mittel & Riesig & Groß \\ 
\hline 
• & • & • & • & • \\ 
\hline 
• & • & • & • & • \\ 
\hline 
• & • & • & • & • \\ 
\hline 
• & • & • & • & • \\ 
\hline 
\end{tabular} 

\subsection*{Entfernung zwischen Sender- und Empfänger-Gerät}



\section*{Warum}


\subsection*{Displaygrößen}


\subsection*{Analoge Patterns}


\subsection*{State of the Art/Gebrauchshistorie}


\subsection*{Checkliste: Entspricht die Interaktion der Definiton eines "Blended Interaction"?}


\section*{Technisches}

\subsection*{Technologien zur Objekterkennung}


\subsection*{Technologien zur Kommunikation}


\subsection*{Technologien zur Bewegungs-/Orientierungsbestimmung}


\subsection*{Prototyp/Lösungsansatz/Code-Snippets/UML-Diagramm}



\section*{Sonstiges}

\subsection*{Autor/en}

\subsection*{Literaturreferenzen}

\subsection*{Abbildungsverzeichnis}

\subsection*{Versionshistorie}

\subsection*{Kommentare}

\subsection*{Offene Fragen}