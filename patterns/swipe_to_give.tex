\documentclass[11pt,a4paper,notitlepage]{article}

\usepackage[utf8]{inputenc}
\usepackage[T1]{fontenc}
\usepackage[german]{babel}
\usepackage{float}
%\usepackage{amsmath}
%\usepackage{amsfonts}
\usepackage{amssymb}
\usepackage{graphicx}
\usepackage{ifthen}
\usepackage{diagbox}
\usepackage[hyphens]{url}
\usepackage{textcomp,gensymb}
\usepackage{makecell}
\usepackage{textpos}
\usepackage{tabularx}
\usepackage{csquotes}
%\usepackage{hyperref}
\usepackage[natbib=true,bibstyle=numeric,backend=bibtexu,citestyle=numeric]{biblatex}
\bibliography{lit.bib} 
\renewcommand\theadalign{cb}
\renewcommand\theadfont{\bfseries}
\renewcommand\theadgape{\Gape[4pt]}

\usepackage{nopageno}

\author{}
\date{}
\title{\name}

% Template für Checkboxen ----
\newcommand{\checkbox}[1]{
\ifx#1\undefined
  $\Box$
\else
  $\boxtimes$  
\fi}

\setlength{\parindent}{0pt}

%----------------------------

\newcommand{\grafischedarstellung}{\jobname_graphical_description.png}

%----------------------------

\newcommand{\umldiagram}{\jobname_uml.png}

%----------------------------

\newcommand{\sequencediagram}{\jobname_sequence.png}

%----------------------------

\newcommand{\solutionimg}{\jobname_solution.png}

%----------------------------

\newcommand{\prototypeimg}{\jobname_prototype.png}

% --------- Glossary
\newcommand{\sen}{Sender}
\newcommand{\rec}{Empfänger}
\newcommand{\recdev}{Empfangsgerät}
\newcommand{\sendev}{Sendegerät}
\newcommand{\data}{Datenobjekt}

\newcommand{\name}{Swipe To Give}

% -------------------------------
% WAS
% -------------------------------

\newcommand{\desc}{Ein Benutzer (der \sen) hat ein \data\ und will es mit einem weiteren Benutzer (dem \rec) bzw. einem \recdev\ teilen. Es soll über Distanz vom \sendev\ auf das \recdev\ übertragen werden, sodass es anschließend auf beiden Geräten verfügbar ist.}

\newcommand{\solution}{Der \sen\ hat das \sendev{} in der Hand oder vor sich liegen. Durch eine Wischbewegung (\Gls{swipe}) in Richtung des \recdev{}s wird das \data{} auf das \recdev{} übertragen.}

\newcommand{\category}{give}
%\newcommand{\category}{take}
%\newcommand{\category}{exchange}
%\newcommand{\category}{extend}
%\newcommand{\category}{connect}

% -------------------------------
% WIE
% -------------------------------

\newcommand{\useraction}{Der \sen\ hält das \sendev\ in der Hand oder hat es vor sich liegen bzw. stehen (z.B. ein Tablet oder Tabletop). Ein \data\ ist entweder explizit vom \sen\ (z.B. ein Foto) oder implizit durch die Applikation (z.B. der aktuelle Bildschirm) zum Versenden ausgewählt worden. \recdev\ und \sendev\ sind bereits zum bidirektionalen Datenaustausch verbunden.\\

Der \sen\ führt einen Swipe mit einem Finger in Richtung des \recdev s\ aus. Die Dynamik der Swipe Geste sollte dem Sender das Gefühl geben, die physikalische Distanz zum \recdev\ durch Bewegung des \data s\ zu überbrücken. Diese Dynamik kann über Anpassung der Parameter (\textit{Länge}, \textit{Richtung}, \textit{Geschwindigkeit}) des Swipes gestaltet werden. Von den gewählten Parametern hängt ab, wann ein Swipe als \textit{erfolgreich} oder \textit{nicht erfolgreich} erkannt wird.}

\newcommand{\reactionSen}{Auf dem \sendev\ kann dem Nutzer zu verschiedenen Phasen des Swipes ein Feedback gegeben werden. Ein Swipe besteht aus den drei \glslink{atomareinteraktion}{atomaren Interaktionen} \textit{Touch}, \textit{Move} und \textit{Release}.\\
Bei einem \textit{erfolgreich} ausgeführten Swipe finden alle atomaren Interaktionen statt, nach Beendigung der letzten atomaren Interaktion wird der Transfer des selektierten \data s\ ausgeführt. Bei \textit{nicht erfolgreichem} Swipe wird kein Transfer ausgeführt, einzelne atomare Interaktionen des Swipes können entfallen.}

\newcommand{\reactionRec}{Auf dem \recdev\ kann dem Nutzer bei erfolgreichem Swipe auf dem \sendev\ und anschließendem erfolgreichen Transfer ein Feedback über den Empfang des \data s\ gegeben werden (Atomare Interaktion \textit{Receive}).}

\newcommand{\microinteractionstabular}{
\begin{figure}[H]
\begin{table}[H]
\renewcommand{\arraystretch}{2}\addtolength{\tabcolsep}{-2pt}
\centering
\newcolumntype{b}{X}
\newcolumntype{t}{>{\hsize=.3\hsize}X}
\newcolumntype{s}{>{\hsize=.2\hsize}c}
\newcolumntype{m}{>{\hsize=.6\hsize}X}
\begin{tabularx}{\textwidth}{tsbbm}
\thead[X]{Name} & \thead[c]{Typ*} & \thead[X]{Trigger} & \thead[X]{Regeln} & \thead[X]{Feedback} \\
\hline
Touch & M & Touch Down Event auf dem Screen & Touch auf dem \data\ &  Animation 1 \\ 
\hline
Move & M & Touch Move Event auf dem Screen & 
Touch ausgeführt, \newline Release nicht ausgeführt & Animation 2 \\ 
\hline
Release & M & Touch Up Event auf dem Screen & Swipelänge OK, \newline Swipedauer OK, \newline Swipeorientierung OK & Animation 3 \\ 
\hline
Receive & S & Daten empfangen & \data\ ist darstellbar & Animation 4 \newline Vibration \\
\hline
\end{tabularx}
\end{table}
\caption{Atomare Interaktionen für das Swipe To Give Pattern}
\end{figure}
*Typ: (M)anuell, (S)ystem
}

\newcommand{\animations}{
\begin{enumerate}
\item Touch-Animation: visualisiert dem Benutzer, ob der Touch auf dem \data\ ausgeführt wurde (z.B. Ripple-Effekt)
\item Move-Animation: visualisiert dem Benutzer, dass das \data\ beweglich ist (z.B. durch Drag-And-Drop-Animation)
\item Release-Animation: visualisiert dem Benutzer, dass das Objekt losgelassen und entweder versendet (\textit{erfolgreicher Swipe}) oder nicht versendet wurde (\textit{nicht erfolgreicher Swipe})
\item Receive-Animation: visualisiert dem Benutzer, dass ein \data\ empfangen wurde
\end{enumerate}
}

\newcommand{\designnotes}{Es sollte darauf geachtet werden, dass die Swipe Geste im Verwendungskontext nicht konkurrierend eingesetzt wird, also z.B. von Betriebssystemfeatures wie einem per Swipe zu öffnenden Einstellungsmenü überlagert wird. Zudem sollte innerhalb des Applikationskontextes keine weitere Funktionalität durch einen Swipe ausgelöst werden (z.B. Scrollen, Umblättern etc.).\\

Der Swipe sollte zur eindeutigen Adressierung in Richtung des \recdev s\ ausgeführt werden. Da es unter Umständen schwierig ist, die Orientierung verschiedener Geräte zueinander zu bestimmen, kann alternativ auch eine feste Richtung für den Swipe vorgegeben werden (z.B. immer von unten nach oben). In dem Fall sollte der Nutzer auf die vorgegebene Richtung visuell hingewiesen werden (z.B. durch eine Markierung am Bildschirmrand).}

% -------------------------------
% WANN
% -------------------------------

\newcommand{\validcontext}{Datenaustausch von Bildern, Videos, Visitenkarten, Social Network IDs, Systembefehlen}

%\newcommand{\simultaneously}{}
\newcommand{\sequentially}{}

\newcommand{\private}{}
\newcommand{\semipublic}{}
\newcommand{\public}{}
%\newcommand{\stationary}{}
\newcommand{\onthego}{}

\newcommand{\leanback}{}
\newcommand{\leanforward}{}

\newcommand{\single}{}
\newcommand{\collaboration}{}
\newcommand{\facetoface}{}
%\newcommand{\sidetoside}{}

\newcommand{\notvalidcontext}{--- keine Information ---}

\newcommand{\devicetabular}{
\begin{figure}[H]
\begin{tabular}{|c|c|c|c|c|c|}
\hline 
\diagbox{von}{nach} & Smartwatch & Smartphone & Tablet & Tabletop & Screens \\ 
\hline 
Smartwatch          &            &     x      &   x    &     x    &     x   \\ 
\hline 
Smartphone          &            &     x      &   x    &     x    &     x   \\ 
\hline 
Tablet              &            &            &   x    &     x    &     x   \\ 
\hline 
Tabletop            &            &            &        &          &     x   \\ 
\hline
Screens             &            &            &        &          &         \\ 
\hline 
\end{tabular}
\caption{Geräteklassen für das Swipe To Give Pattern} 
\end{figure}
}

% -------------------------------
% WARUM
% -------------------------------

\newcommand{\established}{}
%\newcommand{\candidate}{}
%\newcommand{\realizable}{}
%\newcommand{\futuristic}{}

\newcommand{\otherpatterns}{Stitch, Extend With Two Fingers}

\newcommand{\stateoftheart}{
\begin{enumerate}
\item \url{https://www.youtube.com/watch?v=GDdPN6mVLPM} (Audi Deutschland) \\
von Minute 0:56 bis 0:59\\
Mit der Swipe-Geste auf vom Gerät, wird das konfigurierte Auto auf die
Leinwand gegenüber des Kunden projiziert.

\item \url{http://vimeo.com/53606494} (Razorfish - Emerging Experiences) \\
von Minute 2:14 bis 2:25 \\
Mit der Swipe-Geste auf dem Tablet simuliert der Verkäufer (rotes
Hemd) dem Kunden (blaues Hemd) wie diverse Krawatten an ihm aussehen.

\item \url{http://www.microsoft.com/office/vision/} (Microsoft) \\
von Minute 3:41 bis 3:45

\item \url{https://www.youtube.com/watch?v=ho00x7ZvDLw} (Microsoft) \\
von Minute 0:35 bis 0:37

\item \url{https://www.youtube.com/watch?v=o_hKFOQolIg} \\
von Minute 1:04 bis 1:09 und 1:30 bis 1:45 in kollaborativer Nutzung

\item \url{http://vimeo.com/110928116} (Razorfish - Emerging Experiences) \\
von Minute 1:09 bis 1:15

\item \url{https://www.youtube.com/watch?v=yw564p8oF1M} \\
von Minute 0:14 bis 0:20 \\

\item \url{https://www.youtube.com/watch?v=26I90S5PKVU&feature=youtu.be} \\ Werbung von o2 mit Swipe To Give als Bezahlmethode.

\item Verschiedene kommerzielle Applikationen zum Übertragen von Dateien per Swipe-Geste \cite{Fotoswipe} \cite{Sokolov2012} \cite{Hoccer2015}.
\item Beispiel-Implementierung des Swipe To Give Patterns \citep{Hahn2015}.
\item Microinteractions für Swipe-Gesten \cite{Madden2016}
\end{enumerate}
}

\newcommand{\designprinciples}{}

\newcommand{\imageschemata}{}
%\newcommand{\imageSchemaCenterPeriphery}{}
%\newcommand{\imageSchemaContact}{}
%\newcommand{\imageSchemaFrontBack}{}
%\newcommand{\imageSchemaLocation}{}
%\newcommand{\imageSchemaNearFar}{}
\newcommand{\imageSchemaObject}{}
\newcommand{\imageSchemaPath}{}
\newcommand{\imageSchemaSourcePathGoal}{}
%\newcommand{\imageSchemaScale}{}
%\newcommand{\imageSchemaLeftRight}{}
\newcommand{\imageSchemaContainer}{}
\newcommand{\imageSchemaContent}{}
%\newcommand{\imageSchemaFullEmpty}{}
\newcommand{\imageSchemaInOut}{}
%\newcommand{\imageSchemaSurface}{}
%\newcommand{\imageSchemaMerging}{}
%\newcommand{\imageSchemaSplitting}{}
\newcommand{\imageSchemaMomentum}{}
%\newcommand{\imageSchemaSelfMotion}{}
%\newcommand{\imageSchemaBigSmall}{}
\newcommand{\imageSchemaFastSlow}{}
%\newcommand{\imageSchemaPartWhole}{}

\newcommand{\realworld}{}
\newcommand{\realworldNaivePhysic}{}
\newcommand{\realworldBodyAwareness}{}
\newcommand{\realworldEnvironmentAwareness}{}
\newcommand{\realworldSocialAwareness}{}

\newcommand{\metaphor}{}
\newcommand{\metaphordesc}{Jemandem ein Object durch die Luft zuwerfen.}

% -------------------------------
% TECHNISCHES
% -------------------------------

\newcommand{\requiredTechnologies}{
Um Swipe To Give auf einem Gerät (\textit{Device}) einsetzen zu können, gibt es einige Voraussetzungen und Einschränkungen bezüglich der verfügbaren Technologien auf diesem Gerät. Ein Gerät ist dann für das Swipe To Give Pattern verwendbar, wenn es folgende Eigenschaften aufweist:
\begin{itemize}
\item \textbf{Input}: Für die Erkennung der Swipe To Give Geste ist ein Touchscreen notwendig. Bezüglich der technischen Funktionsweise des \textit{Screens} (kapazitiv, resistiv etc.) gibt es keine Einschränkungen, da ein einzelner Touchpunkt von den meisten gängigen Technologien erkannt wird.
\item \textbf{Output}: Je nachdem, welches Feedback dem User gegeben werden soll, sind \textit{Output} per Bildschirm sowie Vibration und Sound denkbar, wobei letztere optional sind.
\item \textbf{Connectivity}: Die Übertragung von Daten beim Swipe To Give erfolgt entweder über ein drahtloses Infrastrukturnetzwerk wie WLAN oder über Ad-Hoc-Netzwerke.
\end{itemize}

Abbildung \ref{swipe_hardware} fasst die benötigten Technologien modellhaft zusammen.

\begin{figure}[h]
\includegraphics[width=\textwidth]{swipe_hardware.png}
\caption{Benötigte Technologien für das Swipe To Give Pattern}
\label{swipe_hardware}
\end{figure}
}

\newcommand{\implementation}{
\subsubsection*{Ablauf Gestenerkennung}
Bei der Geste Swipe To Give handelt es sich um eine \textit{einfache Geste}, deren Erkennung nur auf einem Gerät durchgeführt werden muss. Der allgemeine Ablauf entfällt in zwei Teile (s. Abbildung \ref{gesture_detection}):
\begin{itemize}
\item Erkennen der Geste (\textit{Gesture Detection}) und
\item Überprüfen eventueller Bedingungen an die Geste (\textit{Constraint Check}).
\end{itemize}

\begin{figure}[h]
\includegraphics[width=\textwidth]{gesture_detection.png}
\caption{Allgemeiner Ablauf einer Gestenerkennung}
\label{gesture_detection}
\end{figure}
Wurde die Geste erkannt, wird ein entsprechendes \textit{GestureEvent} generiert, das an den \textit{Constraint Check} übergeben wird.

\subsubsection*{Swipe Erkennung}
Zum Erkennen der Geste wird ein \textit{Touchscreen} (resistiv, kapazitiv o.Ä.) vorausgesetzt. Folgende Events sollten durch den Touchscreen zur Verfügung gestellt werden:
\begin{itemize}
\item \textbf{DOWN}: Berührung des Touchscreens mit einem Finger (mit Koordinaten)
\item \textbf{UP}: Verlassen des Touchscreens mit einem Finger (mit Koordinaten)
\item \textbf{MOVE} [optional]: Weitere Berührungen zwischen \textbf{DOWN} und \textbf{UP}, z.B. für Animationen des Swipes
\end{itemize}

Basierend auf diesen Events ergibt sich die Implementierung der Swipe-Erkennung entsprechend Abbildung \ref{recognize_swipe}. Ein Swipe wird dann erkannt, wenn nach einem initialen \textbf{DOWN}-Event und beliebig vielen \textbf{MOVE}-Events ein \textbf{UP}-Event stattfindet. Die Empfindlichkeit der Swipe-Erkennung kann an dieser Stelle über Constraints gesteuert werden, die bspw. Geschwindigkeit, Dauer oder Richtung des Swipes einschränken. Das Beispiel in Abbildung \ref{recognize_swipe} definiert einen Parameter \textit{min\_velocity}, durch den eine Mindestgeschwindigkeit vorausgesetzt wird. Wurde der Swipe erkannt, wird ein SwipeEvent generiert, das für die Überprüfung der \textit{Constraints} im nächsten Schritt benötigt wird.

\begin{figure}[h]
\includegraphics[width=\textwidth]{swipe_recognize.png}
\caption{Erkennung der Swipe Geste}
\label{recognize_swipe}
\end{figure}

\subsubsection*{Swipe Constraint Check}
Um einzuschränken, unter welchen Bedingungen ein Swipe als Swipe To Give gewertet werden soll, wird das im vorigen Schritt generierte \textit{SwipeEvent} mit vorher gesetzten \textit{Constraints} verglichen. Abbildung \ref{check_constraints} zeigt beispielhaft den Ablauf des \textit{Constraint Checks}. Es werden zwei \textit{Constraints} überprüft:
\begin{itemize}
\item die Richtung des Swipes und
\item die zeitliche Dauer des Swipes.
\end{itemize}
Beide Constraints werden anhand von Variablen überprüft, die aus den \textit{UP}- und \textit{DOWN}-Events der Geste abgeleitet und mit definierten Richtwerten verglichen werden können. In diesem Fall würde ein Swipe To Give ausgelöst, wenn der Swipe zum rechten Bildschirmrand mit einer Mindestdauer von 250ms ausgeführt wird. Beliebige weitere \textit{Constraints} sind denkbar, solange sich die Variablen aus den Rohdaten des Touchscreens ableiten lassen.

\begin{figure}[h]
\includegraphics[width=\textwidth]{swipe_check_constraints.png}
\caption{Überprüfung der Constraints für die Swipe Geste}
\label{check_constraints}
\end{figure}

Entsprechend dem \textit{Lebenszyklus} einer Multiscreen-Applikation wird nach erfolgreich ausgeführtem Swipe ein \textit{Transfer} \textit{selektierter Daten} durchgeführt, eine \textit{Connection} besteht bereits.

Weitergehende Informationen zum Applikations-Lebenszyklus und den weiteren Gestaltungsmöglichkeiten für den \textit{Transfer}, \textit{Feedbacks} etc. finden sich auf der \developerpage.
}

% -------------------------------
% SONSTIGES
% -------------------------------

\newcommand{\authors}{
Alexander Hahn, Hochschule Mannheim  \\
Valentina Burjan, Hochschule Mannheim \\
Dominick Madden, Hochschule Mannheim \\
Horst Schneider, Hochschule Mannheim}

\newcommand{\versionhistory}{10.08.2016}
\newcommand{\dateofcreation}{17.09.2015}
\newcommand{\comments}{---}
\newcommand{\questions}{---}


% template inkludieren --------------

\maketitle


\section*{Was}

\subsection*{Problem}
\desc

\subsection*{Lösung}
\solution

\subsection*{Grafische Darstellung}

\begin{figure}[H]
\includegraphics[scale=0.3]{mypicture.png}
\end{figure}


\subsection*{Kategorie}
\ifthenelse{\equal{\category}{give}}{$\boxtimes$}{$\Box$} Give \\
\ifthenelse{\equal{\category}{take}}{$\boxtimes$}{$\Box$} Take \\
\ifthenelse{\equal{\category}{exchange}}{$\boxtimes$}{$\Box$} Exchange \\
\ifthenelse{\equal{\category}{extend}}{$\boxtimes$}{$\Box$} Extend \\
\ifthenelse{\equal{\category}{connect}}{$\boxtimes$}{$\Box$} Connect



\section*{Wie}

\subsection*{Aktion des Benutzers}
\useraction

\subsection*{Reaktion des Sende-und Empfänger-Gerätes}
%\reaction

\subsection*{Hinweise zur Gestaltung der Interaktion}
%\designnotes



\section*{Wann}

\subsection*{Geeigneter Nutzungskontext}

\subsubsection*{Zeit}
\ifthenelse{\equal{\when}{gleichzeitig}}{$\boxtimes$}{$\Box$} gleichzeitige Nutzung von Geräten \\
\ifthenelse{\equal{\when}{aufeinanderfolgend}}{$\boxtimes$}{$\Box$} sequentielle Nutzung von Geräten 

\subsubsection*{Modus}
\ifthenelse{\equal{\mode}{online}}{$\boxtimes$}{$\Box$} online \\
\ifthenelse{\equal{\mode}{offline}}{$\boxtimes$}{$\Box$} offline \\

%\validcontext

\subsection*{Abzuratender Nutzungskontext}
%\notvalidcontext

\subsection*{Geräteklassen}
\begin{tabular}{|c|c|c|c|c|}
\hline 
• & • & Mittel & Riesig & Groß \\ 
\hline 
• & • & • & • & • \\ 
\hline 
• & • & • & • & • \\ 
\hline 
• & • & • & • & • \\ 
\hline 
• & • & • & • & • \\ 
\hline 
\end{tabular} 

\subsection*{Entfernung zwischen Sender- und Empfänger-Gerät}



\section*{Warum}


\subsection*{Displaygrößen}


\subsection*{Analoge Patterns}


\subsection*{State of the Art/Gebrauchshistorie}


\subsection*{Checkliste: Entspricht die Interaktion der Definiton eines "Blended Interaction"?}


\section*{Technisches}

\subsection*{Technologien zur Objekterkennung}


\subsection*{Technologien zur Kommunikation}


\subsection*{Technologien zur Bewegungs-/Orientierungsbestimmung}


\subsection*{Prototyp/Lösungsansatz/Code-Snippets/UML-Diagramm}



\section*{Sonstiges}

\subsection*{Autor/en}

\subsection*{Literaturreferenzen}

\subsection*{Abbildungsverzeichnis}

\subsection*{Versionshistorie}

\subsection*{Kommentare}

\subsection*{Offene Fragen}